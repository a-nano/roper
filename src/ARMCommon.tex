\documentclass{article}
\usepackage[utf8]{inputenc}
\usepackage{amssymb}
\usepackage{amsmath}
\usepackage{listings}
\let\oldemptyset\emptyset
\let\emptyset\varnothing
\usepackage{parskip}
\usepackage{fancyvrb}

\title{Appendix: Code Used by Both Thumb and ARM Grammars}
\author{Olivia Lucca Fraser\\B00109376}
\date{\today}

%% ODER: format ==         = "\mathrel{==}"
%% ODER: format /=         = "\neq "
%
%
\makeatletter
\@ifundefined{lhs2tex.lhs2tex.sty.read}%
  {\@namedef{lhs2tex.lhs2tex.sty.read}{}%
   \newcommand\SkipToFmtEnd{}%
   \newcommand\EndFmtInput{}%
   \long\def\SkipToFmtEnd#1\EndFmtInput{}%
  }\SkipToFmtEnd

\newcommand\ReadOnlyOnce[1]{\@ifundefined{#1}{\@namedef{#1}{}}\SkipToFmtEnd}
\usepackage{amstext}
\usepackage{amssymb}
\usepackage{stmaryrd}
\DeclareFontFamily{OT1}{cmtex}{}
\DeclareFontShape{OT1}{cmtex}{m}{n}
  {<5><6><7><8>cmtex8
   <9>cmtex9
   <10><10.95><12><14.4><17.28><20.74><24.88>cmtex10}{}
\DeclareFontShape{OT1}{cmtex}{m}{it}
  {<-> ssub * cmtt/m/it}{}
\newcommand{\texfamily}{\fontfamily{cmtex}\selectfont}
\DeclareFontShape{OT1}{cmtt}{bx}{n}
  {<5><6><7><8>cmtt8
   <9>cmbtt9
   <10><10.95><12><14.4><17.28><20.74><24.88>cmbtt10}{}
\DeclareFontShape{OT1}{cmtex}{bx}{n}
  {<-> ssub * cmtt/bx/n}{}
\newcommand{\tex}[1]{\text{\texfamily#1}}	% NEU

\newcommand{\Sp}{\hskip.33334em\relax}


\newcommand{\Conid}[1]{\mathit{#1}}
\newcommand{\Varid}[1]{\mathit{#1}}
\newcommand{\anonymous}{\kern0.06em \vbox{\hrule\@width.5em}}
\newcommand{\plus}{\mathbin{+\!\!\!+}}
\newcommand{\bind}{\mathbin{>\!\!\!>\mkern-6.7mu=}}
\newcommand{\rbind}{\mathbin{=\mkern-6.7mu<\!\!\!<}}% suggested by Neil Mitchell
\newcommand{\sequ}{\mathbin{>\!\!\!>}}
\renewcommand{\leq}{\leqslant}
\renewcommand{\geq}{\geqslant}
\usepackage{polytable}

%mathindent has to be defined
\@ifundefined{mathindent}%
  {\newdimen\mathindent\mathindent\leftmargini}%
  {}%

\def\resethooks{%
  \global\let\SaveRestoreHook\empty
  \global\let\ColumnHook\empty}
\newcommand*{\savecolumns}[1][default]%
  {\g@addto@macro\SaveRestoreHook{\savecolumns[#1]}}
\newcommand*{\restorecolumns}[1][default]%
  {\g@addto@macro\SaveRestoreHook{\restorecolumns[#1]}}
\newcommand*{\aligncolumn}[2]%
  {\g@addto@macro\ColumnHook{\column{#1}{#2}}}

\resethooks

\newcommand{\onelinecommentchars}{\quad-{}- }
\newcommand{\commentbeginchars}{\enskip\{-}
\newcommand{\commentendchars}{-\}\enskip}

\newcommand{\visiblecomments}{%
  \let\onelinecomment=\onelinecommentchars
  \let\commentbegin=\commentbeginchars
  \let\commentend=\commentendchars}

\newcommand{\invisiblecomments}{%
  \let\onelinecomment=\empty
  \let\commentbegin=\empty
  \let\commentend=\empty}

\visiblecomments

\newlength{\blanklineskip}
\setlength{\blanklineskip}{0.66084ex}

\newcommand{\hsindent}[1]{\quad}% default is fixed indentation
\let\hspre\empty
\let\hspost\empty
\newcommand{\NB}{\textbf{NB}}
\newcommand{\Todo}[1]{$\langle$\textbf{To do:}~#1$\rangle$}

\EndFmtInput
\makeatother
%
%
%
%
%
%
% This package provides two environments suitable to take the place
% of hscode, called "plainhscode" and "arrayhscode". 
%
% The plain environment surrounds each code block by vertical space,
% and it uses \abovedisplayskip and \belowdisplayskip to get spacing
% similar to formulas. Note that if these dimensions are changed,
% the spacing around displayed math formulas changes as well.
% All code is indented using \leftskip.
%
% Changed 19.08.2004 to reflect changes in colorcode. Should work with
% CodeGroup.sty.
%
\ReadOnlyOnce{polycode.fmt}%
\makeatletter

\newcommand{\hsnewpar}[1]%
  {{\parskip=0pt\parindent=0pt\par\vskip #1\noindent}}

% can be used, for instance, to redefine the code size, by setting the
% command to \small or something alike
\newcommand{\hscodestyle}{}

% The command \sethscode can be used to switch the code formatting
% behaviour by mapping the hscode environment in the subst directive
% to a new LaTeX environment.

\newcommand{\sethscode}[1]%
  {\expandafter\let\expandafter\hscode\csname #1\endcsname
   \expandafter\let\expandafter\endhscode\csname end#1\endcsname}

% "compatibility" mode restores the non-polycode.fmt layout.

\newenvironment{compathscode}%
  {\par\noindent
   \advance\leftskip\mathindent
   \hscodestyle
   \let\\=\@normalcr
   \let\hspre\(\let\hspost\)%
   \pboxed}%
  {\endpboxed\)%
   \par\noindent
   \ignorespacesafterend}

\newcommand{\compaths}{\sethscode{compathscode}}

% "plain" mode is the proposed default.
% It should now work with \centering.
% This required some changes. The old version
% is still available for reference as oldplainhscode.

\newenvironment{plainhscode}%
  {\hsnewpar\abovedisplayskip
   \advance\leftskip\mathindent
   \hscodestyle
   \let\hspre\(\let\hspost\)%
   \pboxed}%
  {\endpboxed%
   \hsnewpar\belowdisplayskip
   \ignorespacesafterend}

\newenvironment{oldplainhscode}%
  {\hsnewpar\abovedisplayskip
   \advance\leftskip\mathindent
   \hscodestyle
   \let\\=\@normalcr
   \(\pboxed}%
  {\endpboxed\)%
   \hsnewpar\belowdisplayskip
   \ignorespacesafterend}

% Here, we make plainhscode the default environment.

\newcommand{\plainhs}{\sethscode{plainhscode}}
\newcommand{\oldplainhs}{\sethscode{oldplainhscode}}
\plainhs

% The arrayhscode is like plain, but makes use of polytable's
% parray environment which disallows page breaks in code blocks.

\newenvironment{arrayhscode}%
  {\hsnewpar\abovedisplayskip
   \advance\leftskip\mathindent
   \hscodestyle
   \let\\=\@normalcr
   \(\parray}%
  {\endparray\)%
   \hsnewpar\belowdisplayskip
   \ignorespacesafterend}

\newcommand{\arrayhs}{\sethscode{arrayhscode}}

% The mathhscode environment also makes use of polytable's parray 
% environment. It is supposed to be used only inside math mode 
% (I used it to typeset the type rules in my thesis).

\newenvironment{mathhscode}%
  {\parray}{\endparray}

\newcommand{\mathhs}{\sethscode{mathhscode}}

% texths is similar to mathhs, but works in text mode.

\newenvironment{texthscode}%
  {\(\parray}{\endparray\)}

\newcommand{\texths}{\sethscode{texthscode}}

% The framed environment places code in a framed box.

\def\codeframewidth{\arrayrulewidth}
\RequirePackage{calc}

\newenvironment{framedhscode}%
  {\parskip=\abovedisplayskip\par\noindent
   \hscodestyle
   \arrayrulewidth=\codeframewidth
   \tabular{@{}|p{\linewidth-2\arraycolsep-2\arrayrulewidth-2pt}|@{}}%
   \hline\framedhslinecorrect\\{-1.5ex}%
   \let\endoflinesave=\\
   \let\\=\@normalcr
   \(\pboxed}%
  {\endpboxed\)%
   \framedhslinecorrect\endoflinesave{.5ex}\hline
   \endtabular
   \parskip=\belowdisplayskip\par\noindent
   \ignorespacesafterend}

\newcommand{\framedhslinecorrect}[2]%
  {#1[#2]}

\newcommand{\framedhs}{\sethscode{framedhscode}}

% The inlinehscode environment is an experimental environment
% that can be used to typeset displayed code inline.

\newenvironment{inlinehscode}%
  {\(\def\column##1##2{}%
   \let\>\undefined\let\<\undefined\let\\\undefined
   \newcommand\>[1][]{}\newcommand\<[1][]{}\newcommand\\[1][]{}%
   \def\fromto##1##2##3{##3}%
   \def\nextline{}}{\) }%

\newcommand{\inlinehs}{\sethscode{inlinehscode}}

% The joincode environment is a separate environment that
% can be used to surround and thereby connect multiple code
% blocks.

\newenvironment{joincode}%
  {\let\orighscode=\hscode
   \let\origendhscode=\endhscode
   \def\endhscode{\def\hscode{\endgroup\def\@currenvir{hscode}\\}\begingroup}
   %\let\SaveRestoreHook=\empty
   %\let\ColumnHook=\empty
   %\let\resethooks=\empty
   \orighscode\def\hscode{\endgroup\def\@currenvir{hscode}}}%
  {\origendhscode
   \global\let\hscode=\orighscode
   \global\let\endhscode=\origendhscode}%

\makeatother
\EndFmtInput
%

\begin{document}

\maketitle

\section{Imports}

There is a considerable amount of code that can be reused between the
Thumb and ARM modules, which I have gathered together in this file.

First, let's begin by importing a few modules.

\begin{hscode}\SaveRestoreHook
\column{B}{@{}>{\hspre}l<{\hspost}@{}}%
\column{E}{@{}>{\hspre}l<{\hspost}@{}}%
\>[B]{}\mathbf{module}\;\Conid{ARMCommon}\;\mathbf{where}{}\<[E]%
\\[\blanklineskip]%
\>[B]{}\mathbf{import}\;\Conid{Aux}{}\<[E]%
\\
\>[B]{}\mathbf{import}\;\Conid{\Conid{Data}.Word}{}\<[E]%
\\
\>[B]{}\mathbf{import}\;\Conid{\Conid{Data}.Bits}{}\<[E]%
\\
\>[B]{}\mathbf{import}\;\Conid{\Conid{Data}.List}{}\<[E]%
\\
\>[B]{}\mathbf{import}\;\Conid{\Conid{Data}.Tuple}{}\<[E]%
\\
\>[B]{}\mathbf{import}\;\Conid{\Conid{Data}.Maybe}{}\<[E]%
\ColumnHook
\end{hscode}\resethooks

\section{Types}

Now, let's define some types that we'll be using in each module.
Regardless of whether we're using the ARM or Thumb modes, the size
of the registers remains constant at 32.\footnote{We are sticking to the 32-bit ARM-7 specifications in this parser, though a 64-bit model of the ARM also exists.}

\begin{hscode}\SaveRestoreHook
\column{B}{@{}>{\hspre}l<{\hspost}@{}}%
\column{18}{@{}>{\hspre}l<{\hspost}@{}}%
\column{23}{@{}>{\hspre}l<{\hspost}@{}}%
\column{E}{@{}>{\hspre}l<{\hspost}@{}}%
\>[B]{}\mathbf{type}\;\Conid{Register}\mathrel{=}\Conid{Word32}{}\<[E]%
\\
\>[B]{}\Varid{registerSize}\mathrel{=}\mathrm{32}{}\<[E]%
\\
\>[B]{}\mathbf{type}\;\Conid{CPSR}\mathrel{=}\Conid{Register}{}\<[E]%
\\
\>[B]{}\mathbf{type}\;\Conid{Operation}{}\<[18]%
\>[18]{}\mathrel{=}(\Conid{CPSR}\to \Conid{Register}\to \Conid{Register}\to \Conid{Register}{}\<[E]%
\\
\>[18]{}\hsindent{5}{}\<[23]%
\>[23]{}\to (\Conid{CPSR},\Conid{Register})){}\<[E]%
\ColumnHook
\end{hscode}\resethooks

The same register set is used in both Thumb and ARM mode, so we'll keep
the same register contents in both.

\begin{hscode}\SaveRestoreHook
\column{B}{@{}>{\hspre}l<{\hspost}@{}}%
\column{E}{@{}>{\hspre}l<{\hspost}@{}}%
\>[B]{}\Varid{fp}\mathrel{=}\mathrm{11}\mathbin{::}\Conid{Int}{}\<[E]%
\\
\>[B]{}\Varid{sp}\mathrel{=}\mathrm{13}\mathbin{::}\Conid{Int}{}\<[E]%
\\
\>[B]{}\Varid{lr}\mathrel{=}\mathrm{14}\mathbin{::}\Conid{Int}{}\<[E]%
\\
\>[B]{}\Varid{pc}\mathrel{=}\mathrm{15}\mathbin{::}\Conid{Int}{}\<[E]%
\\[\blanklineskip]%
\>[B]{}\Varid{highbit}\mathbin{::}\Conid{Register}\to \Conid{Register}{}\<[E]%
\\
\>[B]{}\Varid{highbit}\mathrel{=}(\mathbin{.\&.}\Varid{bit}\;(\Varid{registerSize}\mathbin{-}\mathrm{1})){}\<[E]%
\ColumnHook
\end{hscode}\resethooks


The CPSR (flag register) mechanics are the same in both ARM and Thumb
mode, so let's put everything pertaining to the CPSR here.

A few words on the flags:

\begin{itemize}
  \item The M flags are meant to be read together, as a nibble (a 4-bit word),
        and determine the processor mode.
          \begin{tabular}{l \ensuremath{\Varid{l}} l}
          Bits & Mode & Register Set \\
          0000 & User & R0-R14, CPSR, PC \\
          0001 & FIQ (fast interrupt request)  & R0-R7, R8_fiq-R14_fiq, CPSR, SPSR_fiq, PC \\
          0010 & IRQ & R0-R12, R13_irq, CPSR, SPSR_irq, PC \\
          0011 & SVC (supervisor) & R0-R12, R13_abt-R14_abt, CPSR, SPSR_abt, PC \\
          0111 & ABT (abort) & R0-R12, R13_abt-R14_abt, CPSR, SPSR_abt, PC \\
          1101 & HYP (hypervisor, when supported) & ? \\ %%% Not sure. fill in.
          1011 & UND (undefined) & R0-R12, R13_und-R14_und, CPSR, SPSR_und, PC \\
          1111 & SYS (system) & R0-R14, CPSR, PC \\
          \end{tabular} %
        User mode is the only mode in this list with restricted privileges, and
        the only which cannot freely switch to another mode of its own accord.
        The others are reserved for kernel and, where supported, hypervisor
        operations, and enjoy escalated privileges (analogous to Ring 0 and Ring
        -1 on x86/x86_64 architectures).
  \item The T flag is a one-bit flag that indicates whether to parse and execute
        instructions in 16-bit Thumb mode (1) or 32-bit ARM mode (0).  
\footnote{Following \url{www.heyrick.co.uk/armwiki/The_Status_Register}, supplementing with information from \url{www.keil.com/support/man/docs/armasm/armasm_dom1359731126962.htm}, and Ch. 2 of Dang et al, \emph{Practical Reverse Engineering}, [BIBLIO INOFO].}

\end{itemize}

There is another register called the "saved processor status register",
(SPSR) whose layout varies by mode, and which is used in the various
non-user processor modes. We won't be dealing with that just yet.

\begin{hscode}\SaveRestoreHook
\column{B}{@{}>{\hspre}l<{\hspost}@{}}%
\column{3}{@{}>{\hspre}l<{\hspost}@{}}%
\column{4}{@{}>{\hspre}l<{\hspost}@{}}%
\column{5}{@{}>{\hspre}l<{\hspost}@{}}%
\column{12}{@{}>{\hspre}l<{\hspost}@{}}%
\column{13}{@{}>{\hspre}l<{\hspost}@{}}%
\column{26}{@{}>{\hspre}l<{\hspost}@{}}%
\column{E}{@{}>{\hspre}l<{\hspost}@{}}%
\>[B]{}\mathbf{data}\;\Conid{Flag}\mathrel{=}{}\<[E]%
\\
\>[B]{}\hsindent{5}{}\<[5]%
\>[5]{}\Conid{Tflag}{}\<[13]%
\>[13]{}\mbox{\onelinecomment  enables Thumb mode}{}\<[E]%
\\
\>[B]{}\hsindent{3}{}\<[3]%
\>[3]{}\mid \Conid{Fflag}{}\<[13]%
\>[13]{}\mbox{\onelinecomment  disables FIQ interrupts}{}\<[E]%
\\
\>[B]{}\hsindent{3}{}\<[3]%
\>[3]{}\mid \Conid{Iflag}{}\<[13]%
\>[13]{}\mbox{\onelinecomment  disables IRQ interrupts}{}\<[E]%
\\
\>[B]{}\hsindent{3}{}\<[3]%
\>[3]{}\mid \Conid{Aflag}{}\<[13]%
\>[13]{}\mbox{\onelinecomment  disables imprecise aborts}{}\<[E]%
\\
\>[B]{}\hsindent{3}{}\<[3]%
\>[3]{}\mid \Conid{Jflag}{}\<[13]%
\>[13]{}\mbox{\onelinecomment  enables Jazelle mode (natively run Java bytecode)}{}\<[E]%
\\
\>[B]{}\hsindent{3}{}\<[3]%
\>[3]{}\mid \Conid{Zflag}{}\<[13]%
\>[13]{}\mbox{\onelinecomment  Zero condition}{}\<[E]%
\\
\>[B]{}\hsindent{3}{}\<[3]%
\>[3]{}\mid \Conid{Cflag}{}\<[13]%
\>[13]{}\mbox{\onelinecomment  Carry bit}{}\<[E]%
\\
\>[B]{}\hsindent{3}{}\<[3]%
\>[3]{}\mid \Conid{Qflag}{}\<[13]%
\>[13]{}\mbox{\onelinecomment  underflow or saturation (in E-variants of ARM)}{}\<[E]%
\\
\>[B]{}\hsindent{3}{}\<[3]%
\>[3]{}\mid \Conid{Vflag}{}\<[13]%
\>[13]{}\mbox{\onelinecomment  Overflow condition}{}\<[E]%
\\
\>[B]{}\hsindent{3}{}\<[3]%
\>[3]{}\mid \Conid{Nflag}{}\<[13]%
\>[13]{}\mbox{\onelinecomment  Negative condition}{}\<[E]%
\\
\>[B]{}\hsindent{3}{}\<[3]%
\>[3]{}\mathbf{deriving}\;(\Conid{Eq},\Conid{Show}){}\<[E]%
\\
\>[B]{}\mathbf{instance}\;\Conid{Enum}\;\Conid{Flag}\;\mathbf{where}{}\<[E]%
\\
\>[B]{}\hsindent{3}{}\<[3]%
\>[3]{}\Varid{fromEnum}\mathrel{=}\Varid{fromJust}\mathbin{\circ}\Varid{flip}\;\Varid{lookup}\;\Varid{flagTable}{}\<[E]%
\\
\>[B]{}\hsindent{3}{}\<[3]%
\>[3]{}\Varid{toEnum}{}\<[12]%
\>[12]{}\mathrel{=}\Varid{fromJust}\mathbin{\circ}\Varid{flip}\;\Varid{lookup}\;(\Varid{map}\;\Varid{swap}\;\Varid{flagTable}){}\<[E]%
\\
\>[B]{}\Varid{flagTable}\mathrel{=}[\mskip1.5mu {}\<[E]%
\\
\>[B]{}\hsindent{4}{}\<[4]%
\>[4]{}(\Conid{Tflag},\mathrm{5}){}\<[E]%
\\
\>[B]{}\hsindent{3}{}\<[3]%
\>[3]{},(\Conid{Fflag},\mathrm{6}){}\<[E]%
\\
\>[B]{}\hsindent{3}{}\<[3]%
\>[3]{},(\Conid{Iflag},\mathrm{7}){}\<[E]%
\\
\>[B]{}\hsindent{3}{}\<[3]%
\>[3]{},(\Conid{Aflag},\mathrm{8}){}\<[E]%
\\
\>[B]{}\hsindent{3}{}\<[3]%
\>[3]{},(\Conid{Jflag},\mathrm{24}){}\<[E]%
\\
\>[B]{}\hsindent{3}{}\<[3]%
\>[3]{},(\Conid{Qflag},\mathrm{27}){}\<[E]%
\\
\>[B]{}\hsindent{3}{}\<[3]%
\>[3]{},(\Conid{Vflag},\mathrm{28}){}\<[E]%
\\
\>[B]{}\hsindent{3}{}\<[3]%
\>[3]{},(\Conid{Cflag},\mathrm{29}){}\<[E]%
\\
\>[B]{}\hsindent{3}{}\<[3]%
\>[3]{},(\Conid{Zflag},\mathrm{30}){}\<[E]%
\\
\>[B]{}\hsindent{3}{}\<[3]%
\>[3]{},(\Conid{Nflag},\mathrm{31}){}\<[E]%
\\
\>[B]{}\hsindent{3}{}\<[3]%
\>[3]{}\mskip1.5mu]{}\<[E]%
\\[\blanklineskip]%
\>[B]{}\mathbf{data}\;\Conid{Cond}\mathrel{=}\Conid{C\char95 EQ}\mid \Conid{C\char95 NE}\mid \Conid{C\char95 CS}\mid \Conid{C\char95 CC}\mid \Conid{C\char95 MI}\mid \Conid{C\char95 PL}\mid \Conid{C\char95 VS}\mid \Conid{C\char95 VC}\mid {}\<[E]%
\\
\>[B]{}\hsindent{13}{}\<[13]%
\>[13]{}\Conid{C\char95 HI}\mid \Conid{C\char95 LS}\mid \Conid{C\char95 GE}\mid \Conid{C\char95 LT}\mid \Conid{C\char95 GT}\mid \Conid{C\char95 LE}\mid \Conid{C\char95 AL}\mid {}\<[E]%
\\
\>[B]{}\hsindent{13}{}\<[13]%
\>[13]{}\Conid{C\char95 RESERVED}\;\mathbf{deriving}\;(\Conid{Show},\Conid{Enum}){}\<[E]%
\\[\blanklineskip]%
\>[B]{}\mbox{\onelinecomment  take a list of flags and pack them into a 32-bit bitmap}{}\<[E]%
\\
\>[B]{}\mbox{\onelinecomment  (the CPSR register)}{}\<[E]%
\\
\>[B]{}\Varid{aspr}\mathbin{::}[\mskip1.5mu \Conid{Flag}\mskip1.5mu]\to \Conid{CPSR}{}\<[E]%
\\
\>[B]{}\Varid{aspr}\mathrel{=}\Varid{foldl'}\;(\mathbin{.|.})\;\mathrm{0}\mathbin{\circ}\Varid{map}\;(\Varid{bit}\mathbin{\circ}\Varid{fromEnum}){}\<[E]%
\\[\blanklineskip]%
\>[B]{}\Varid{testFlag}\mathbin{::}\Conid{CPSR}\to \Conid{Flag}\to \Conid{Bool}{}\<[E]%
\\
\>[B]{}\Varid{testFlag}\;\Varid{r}\;\Varid{f}\mathrel{=}\Varid{testBit}\;\Varid{r}\;(\Varid{fromEnum}\;\Varid{f}){}\<[E]%
\\[\blanklineskip]%
\>[B]{}\mbox{\onelinecomment   Return Negative? and Zero? flags if they need to be set}{}\<[E]%
\\
\>[B]{}\mbox{\onelinecomment   based on the value in Register r}{}\<[E]%
\\
\>[B]{}\Varid{setNZFlags}\mathbin{::}\Conid{Register}\to [\mskip1.5mu \Conid{Flag}\mskip1.5mu]{}\<[E]%
\\
\>[B]{}\Varid{setNZFlags}\;\Varid{r}\mathrel{=}\mathbf{let}\;\Varid{res}\mathrel{=}\Varid{fromIntegral}\;\Varid{r}\;\mathbf{in}{}\<[E]%
\\
\>[B]{}\hsindent{3}{}\<[3]%
\>[3]{}[\mskip1.5mu \mskip1.5mu]{}\<[E]%
\\
\>[B]{}\hsindent{3}{}\<[3]%
\>[3]{}\plus \mathbf{if}\;(\Varid{res}\mathbin{<}\mathrm{0})\;\mathbf{then}\;[\mskip1.5mu \Conid{Nflag}\mskip1.5mu]\;\mathbf{else}\;[\mskip1.5mu \mskip1.5mu]{}\<[E]%
\\
\>[B]{}\hsindent{3}{}\<[3]%
\>[3]{}\plus \mathbf{if}\;(\Varid{res}\equiv \mathrm{0})\;\mathbf{then}\;[\mskip1.5mu \Conid{Zflag}\mskip1.5mu]\;\mathbf{else}\;[\mskip1.5mu \mskip1.5mu]{}\<[E]%
\\[\blanklineskip]%
\>[B]{}\mbox{\onelinecomment   Return True if the result of an add (or sub) operation res}{}\<[E]%
\\
\>[B]{}\mbox{\onelinecomment   indicates an overflow with respect to its operands x and y}{}\<[E]%
\\
\>[B]{}\mbox{\onelinecomment   (This is done by testing the high bit in each.)}{}\<[E]%
\\
\>[B]{}\Varid{overflow}\mathbin{::}\Conid{Register}\to \Conid{Register}\to \Conid{Register}\to \Conid{Bool}{}\<[E]%
\\
\>[B]{}\Varid{overflow}\;\Varid{x}\;\Varid{y}\;\Varid{res}\mathrel{=}{}\<[E]%
\\
\>[B]{}\hsindent{3}{}\<[3]%
\>[3]{}\Varid{hx}\equiv \Varid{hy}\mathrel{\wedge}\Varid{hy}\not\equiv \Varid{hr}{}\<[E]%
\\
\>[B]{}\hsindent{3}{}\<[3]%
\>[3]{}\mathbf{where}\;[\mskip1.5mu \Varid{hx},\Varid{hy},\Varid{hr}\mskip1.5mu]\mathrel{=}\Varid{map}\;\Varid{highbit}\;[\mskip1.5mu \Varid{x},\Varid{y},\Varid{res}\mskip1.5mu]{}\<[E]%
\\[\blanklineskip]%
\>[B]{}\mbox{\onelinecomment   It would be nice to have a type class that lets us}{}\<[E]%
\\
\>[B]{}\mbox{\onelinecomment   deal with mnemonics in general. Let's do that.}{}\<[E]%
\\
\>[B]{}\mathbf{class}\;(\Conid{Enum}\;\Varid{a},\Conid{Eq}\;\Varid{a},\Conid{Show}\;\Varid{a})\Rightarrow \Conid{Mnemonic}\;\Varid{a}\;\mathbf{where}{}\<[E]%
\\
\>[B]{}\hsindent{3}{}\<[3]%
\>[3]{}\Varid{additive}\mathbin{::}\Varid{a}\to \Conid{Bool}{}\<[E]%
\\
\>[B]{}\hsindent{3}{}\<[3]%
\>[3]{}\Varid{additive}\;\anonymous \mathrel{=}\Conid{False}{}\<[E]%
\\
\>[B]{}\hsindent{3}{}\<[3]%
\>[3]{}\Varid{writeR}{}\<[12]%
\>[12]{}\mathbin{::}\Varid{a}\to \Conid{Bool}{}\<[E]%
\\
\>[B]{}\hsindent{3}{}\<[3]%
\>[3]{}\Varid{writeR}\;{}\<[12]%
\>[12]{}\anonymous \mathrel{=}\Conid{True}{}\<[26]%
\>[26]{}\mbox{\onelinecomment  True if the operation writes to register, False otherwise}{}\<[E]%
\\
\>[B]{}\hsindent{3}{}\<[3]%
\>[3]{}\Varid{setCPSR}{}\<[12]%
\>[12]{}\mathbin{::}\Varid{a}\to \Conid{Bool}{}\<[E]%
\\
\>[B]{}\hsindent{3}{}\<[3]%
\>[3]{}\Varid{setCPSR}\;{}\<[12]%
\>[12]{}\anonymous \mathrel{=}\Conid{True}{}\<[E]%
\\
\>[B]{}\hsindent{3}{}\<[3]%
\>[3]{}\Varid{ctrlFlow}\mathbin{::}\Varid{a}\to \Conid{Bool}{}\<[E]%
\\
\>[B]{}\hsindent{3}{}\<[3]%
\>[3]{}\Varid{ctrlFlow}\;\anonymous \mathrel{=}\Conid{False}{}\<[E]%
\\[\blanklineskip]%
\>[B]{}\mathbf{class}\;(\Conid{Eq}\;\Varid{a},\Conid{Show}\;\Varid{a})\Rightarrow \Conid{Format}\;\Varid{a}{}\<[E]%
\ColumnHook
\end{hscode}\resethooks

\section{Higher Order Functions for Manipulating Operation Instances}

As we've seen, the signature of the Operation type is a slightly awkward and ungainly thing, and so it would be convenient to have a few auxiliary functions that will help us smoothly compose and recombine these operations without having to tend too much to their internal structure. The following functions represent the beginning of a solution to this problem.

\begin{hscode}\SaveRestoreHook
\column{B}{@{}>{\hspre}l<{\hspost}@{}}%
\column{3}{@{}>{\hspre}l<{\hspost}@{}}%
\column{11}{@{}>{\hspre}l<{\hspost}@{}}%
\column{E}{@{}>{\hspre}l<{\hspost}@{}}%
\>[B]{}\mbox{\onelinecomment   carry over the APSR from the first op to the}{}\<[E]%
\\
\>[B]{}\mbox{\onelinecomment   second, and take the dst reg of the first op}{}\<[E]%
\\
\>[B]{}\mbox{\onelinecomment   as the x (first operand) of the second op}{}\<[E]%
\\
\>[B]{}(\mathbin{@<})\mathbin{::}\Conid{Operation}\to \Conid{Operation}\to \Conid{Operation}{}\<[E]%
\\
\>[B]{}\Varid{op2}\mathbin{@<}\Varid{op1}\mathrel{=}\Varid{compL}{}\<[E]%
\\
\>[B]{}\hsindent{3}{}\<[3]%
\>[3]{}\mathbf{where}\;\Varid{compL}\;\Varid{a1}\;\Varid{x}\;\Varid{y}\;\Varid{d}\mathrel{=}(\Varid{op2}\;\Varid{a2}\;\Varid{d2}\;\Varid{y}\;\Varid{d}){}\<[E]%
\\
\>[3]{}\hsindent{8}{}\<[11]%
\>[11]{}\mathbf{where}\;(\Varid{a2},\Varid{d2})\mathrel{=}\Varid{op1}\;\Varid{a1}\;\Varid{x}\;\Varid{y}\;\Varid{d}{}\<[E]%
\\[\blanklineskip]%
\>[B]{}\mbox{\onelinecomment   carry over the APSR from the first op to the}{}\<[E]%
\\
\>[B]{}\mbox{\onelinecomment   second, and take the dst reg of the first op}{}\<[E]%
\\
\>[B]{}\mbox{\onelinecomment   as y (second operand) for the second op}{}\<[E]%
\\
\>[B]{}(\mathbin{@>})\mathbin{::}\Conid{Operation}\to \Conid{Operation}\to \Conid{Operation}{}\<[E]%
\\
\>[B]{}\Varid{op2}\mathbin{@>}\Varid{op1}\mathrel{=}\Varid{compR}{}\<[E]%
\\
\>[B]{}\hsindent{3}{}\<[3]%
\>[3]{}\mathbf{where}\;\Varid{compR}\;\Varid{a1}\;\Varid{x}\;\Varid{y}\;\Varid{d}\mathrel{=}(\Varid{op2}\;\Varid{a2}\;\Varid{x}\;\Varid{d2}\;\Varid{d}){}\<[E]%
\\
\>[3]{}\hsindent{8}{}\<[11]%
\>[11]{}\mathbf{where}\;(\Varid{a2},\Varid{d2})\mathrel{=}\Varid{op1}\;\Varid{a1}\;\Varid{x}\;\Varid{y}\;\Varid{d}{}\<[E]%
\\[\blanklineskip]%
\>[B]{}\mbox{\onelinecomment   carry over the APSR from the first op to the}{}\<[E]%
\\
\>[B]{}\mbox{\onelinecomment   second, and take the dst reg of the first op}{}\<[E]%
\\
\>[B]{}\mbox{\onelinecomment   as x and y for the second op}{}\<[E]%
\\
\>[B]{}(\mathbin{@<>})\mathbin{::}\Conid{Operation}\to \Conid{Operation}\to \Conid{Operation}{}\<[E]%
\\
\>[B]{}\Varid{op2}\mathbin{@<>}\Varid{op1}\mathrel{=}\Varid{comp}{}\<[E]%
\\
\>[B]{}\hsindent{3}{}\<[3]%
\>[3]{}\mathbf{where}\;\Varid{comp}\;\Varid{a1}\;\Varid{x}\;\Varid{y}\;\Varid{d}\mathrel{=}(\Varid{op2}\;\Varid{a2}\;\Varid{d2}\;\Varid{d2}\;\Varid{d}){}\<[E]%
\\
\>[3]{}\hsindent{8}{}\<[11]%
\>[11]{}\mathbf{where}\;(\Varid{a2},\Varid{d2})\mathrel{=}\Varid{op1}\;\Varid{a1}\;\Varid{x}\;\Varid{y}\;\Varid{d}{}\<[E]%
\\[\blanklineskip]%
\>[B]{}\mbox{\onelinecomment  It would probably make sense to define a state monad or}{}\<[E]%
\\
\>[B]{}\mbox{\onelinecomment  applicative that carries the APSR information across the}{}\<[E]%
\\
\>[B]{}\mbox{\onelinecomment  operations, and lets us use them as ordinary binary ops.}{}\<[E]%
\\
\>[B]{}\mbox{\onelinecomment  This is on the TODO list.}{}\<[E]%
\\[\blanklineskip]%
\>[B]{}\mbox{\onelinecomment  you could also maintain the entire cpu context as a}{}\<[E]%
\\
\>[B]{}\mbox{\onelinecomment  state monad's state. But that's getting a bit heavy handed,}{}\<[E]%
\\
\>[B]{}\mbox{\onelinecomment  and we have an emulator for that. }{}\<[E]%
\\[\blanklineskip]%
\>[B]{}\mbox{\onelinecomment  Now, we need to actually fill in the Operand 1, Operand 2, Dst}{}\<[E]%
\\
\>[B]{}\mbox{\onelinecomment  fields. Maybe this information should be stowed up in the}{}\<[E]%
\\
\>[B]{}\mbox{\onelinecomment  inst record, for easy access. Or perhaps called from an}{}\<[E]%
\\
\>[B]{}\mbox{\onelinecomment  "apply operation" function that populates the operand fields}{}\<[E]%
\\
\>[B]{}\mbox{\onelinecomment  and executes the operation function.}{}\<[E]%
\ColumnHook
\end{hscode}\resethooks

\end{document}
