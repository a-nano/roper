\documentclass{article}
\usepackage[utf8]{inputenc}
\usepackage{amssymb}
\usepackage{amsmath}
\usepackage{listings}
\let\oldemptyset\emptyset
\let\emptyset\varnothing
\usepackage{parskip}
\usepackage{fancyvrb}

\title{ARM Instruction Grammar}
\author{Olivia Lucca Fraser\\B00109376}
\date{\today}

%% ODER: format ==         = "\mathrel{==}"
%% ODER: format /=         = "\neq "
%
%
\makeatletter
\@ifundefined{lhs2tex.lhs2tex.sty.read}%
  {\@namedef{lhs2tex.lhs2tex.sty.read}{}%
   \newcommand\SkipToFmtEnd{}%
   \newcommand\EndFmtInput{}%
   \long\def\SkipToFmtEnd#1\EndFmtInput{}%
  }\SkipToFmtEnd

\newcommand\ReadOnlyOnce[1]{\@ifundefined{#1}{\@namedef{#1}{}}\SkipToFmtEnd}
\usepackage{amstext}
\usepackage{amssymb}
\usepackage{stmaryrd}
\DeclareFontFamily{OT1}{cmtex}{}
\DeclareFontShape{OT1}{cmtex}{m}{n}
  {<5><6><7><8>cmtex8
   <9>cmtex9
   <10><10.95><12><14.4><17.28><20.74><24.88>cmtex10}{}
\DeclareFontShape{OT1}{cmtex}{m}{it}
  {<-> ssub * cmtt/m/it}{}
\newcommand{\texfamily}{\fontfamily{cmtex}\selectfont}
\DeclareFontShape{OT1}{cmtt}{bx}{n}
  {<5><6><7><8>cmtt8
   <9>cmbtt9
   <10><10.95><12><14.4><17.28><20.74><24.88>cmbtt10}{}
\DeclareFontShape{OT1}{cmtex}{bx}{n}
  {<-> ssub * cmtt/bx/n}{}
\newcommand{\tex}[1]{\text{\texfamily#1}}	% NEU

\newcommand{\Sp}{\hskip.33334em\relax}


\newcommand{\Conid}[1]{\mathit{#1}}
\newcommand{\Varid}[1]{\mathit{#1}}
\newcommand{\anonymous}{\kern0.06em \vbox{\hrule\@width.5em}}
\newcommand{\plus}{\mathbin{+\!\!\!+}}
\newcommand{\bind}{\mathbin{>\!\!\!>\mkern-6.7mu=}}
\newcommand{\rbind}{\mathbin{=\mkern-6.7mu<\!\!\!<}}% suggested by Neil Mitchell
\newcommand{\sequ}{\mathbin{>\!\!\!>}}
\renewcommand{\leq}{\leqslant}
\renewcommand{\geq}{\geqslant}
\usepackage{polytable}

%mathindent has to be defined
\@ifundefined{mathindent}%
  {\newdimen\mathindent\mathindent\leftmargini}%
  {}%

\def\resethooks{%
  \global\let\SaveRestoreHook\empty
  \global\let\ColumnHook\empty}
\newcommand*{\savecolumns}[1][default]%
  {\g@addto@macro\SaveRestoreHook{\savecolumns[#1]}}
\newcommand*{\restorecolumns}[1][default]%
  {\g@addto@macro\SaveRestoreHook{\restorecolumns[#1]}}
\newcommand*{\aligncolumn}[2]%
  {\g@addto@macro\ColumnHook{\column{#1}{#2}}}

\resethooks

\newcommand{\onelinecommentchars}{\quad-{}- }
\newcommand{\commentbeginchars}{\enskip\{-}
\newcommand{\commentendchars}{-\}\enskip}

\newcommand{\visiblecomments}{%
  \let\onelinecomment=\onelinecommentchars
  \let\commentbegin=\commentbeginchars
  \let\commentend=\commentendchars}

\newcommand{\invisiblecomments}{%
  \let\onelinecomment=\empty
  \let\commentbegin=\empty
  \let\commentend=\empty}

\visiblecomments

\newlength{\blanklineskip}
\setlength{\blanklineskip}{0.66084ex}

\newcommand{\hsindent}[1]{\quad}% default is fixed indentation
\let\hspre\empty
\let\hspost\empty
\newcommand{\NB}{\textbf{NB}}
\newcommand{\Todo}[1]{$\langle$\textbf{To do:}~#1$\rangle$}

\EndFmtInput
\makeatother
%
%
%
%
%
%
% This package provides two environments suitable to take the place
% of hscode, called "plainhscode" and "arrayhscode". 
%
% The plain environment surrounds each code block by vertical space,
% and it uses \abovedisplayskip and \belowdisplayskip to get spacing
% similar to formulas. Note that if these dimensions are changed,
% the spacing around displayed math formulas changes as well.
% All code is indented using \leftskip.
%
% Changed 19.08.2004 to reflect changes in colorcode. Should work with
% CodeGroup.sty.
%
\ReadOnlyOnce{polycode.fmt}%
\makeatletter

\newcommand{\hsnewpar}[1]%
  {{\parskip=0pt\parindent=0pt\par\vskip #1\noindent}}

% can be used, for instance, to redefine the code size, by setting the
% command to \small or something alike
\newcommand{\hscodestyle}{}

% The command \sethscode can be used to switch the code formatting
% behaviour by mapping the hscode environment in the subst directive
% to a new LaTeX environment.

\newcommand{\sethscode}[1]%
  {\expandafter\let\expandafter\hscode\csname #1\endcsname
   \expandafter\let\expandafter\endhscode\csname end#1\endcsname}

% "compatibility" mode restores the non-polycode.fmt layout.

\newenvironment{compathscode}%
  {\par\noindent
   \advance\leftskip\mathindent
   \hscodestyle
   \let\\=\@normalcr
   \let\hspre\(\let\hspost\)%
   \pboxed}%
  {\endpboxed\)%
   \par\noindent
   \ignorespacesafterend}

\newcommand{\compaths}{\sethscode{compathscode}}

% "plain" mode is the proposed default.
% It should now work with \centering.
% This required some changes. The old version
% is still available for reference as oldplainhscode.

\newenvironment{plainhscode}%
  {\hsnewpar\abovedisplayskip
   \advance\leftskip\mathindent
   \hscodestyle
   \let\hspre\(\let\hspost\)%
   \pboxed}%
  {\endpboxed%
   \hsnewpar\belowdisplayskip
   \ignorespacesafterend}

\newenvironment{oldplainhscode}%
  {\hsnewpar\abovedisplayskip
   \advance\leftskip\mathindent
   \hscodestyle
   \let\\=\@normalcr
   \(\pboxed}%
  {\endpboxed\)%
   \hsnewpar\belowdisplayskip
   \ignorespacesafterend}

% Here, we make plainhscode the default environment.

\newcommand{\plainhs}{\sethscode{plainhscode}}
\newcommand{\oldplainhs}{\sethscode{oldplainhscode}}
\plainhs

% The arrayhscode is like plain, but makes use of polytable's
% parray environment which disallows page breaks in code blocks.

\newenvironment{arrayhscode}%
  {\hsnewpar\abovedisplayskip
   \advance\leftskip\mathindent
   \hscodestyle
   \let\\=\@normalcr
   \(\parray}%
  {\endparray\)%
   \hsnewpar\belowdisplayskip
   \ignorespacesafterend}

\newcommand{\arrayhs}{\sethscode{arrayhscode}}

% The mathhscode environment also makes use of polytable's parray 
% environment. It is supposed to be used only inside math mode 
% (I used it to typeset the type rules in my thesis).

\newenvironment{mathhscode}%
  {\parray}{\endparray}

\newcommand{\mathhs}{\sethscode{mathhscode}}

% texths is similar to mathhs, but works in text mode.

\newenvironment{texthscode}%
  {\(\parray}{\endparray\)}

\newcommand{\texths}{\sethscode{texthscode}}

% The framed environment places code in a framed box.

\def\codeframewidth{\arrayrulewidth}
\RequirePackage{calc}

\newenvironment{framedhscode}%
  {\parskip=\abovedisplayskip\par\noindent
   \hscodestyle
   \arrayrulewidth=\codeframewidth
   \tabular{@{}|p{\linewidth-2\arraycolsep-2\arrayrulewidth-2pt}|@{}}%
   \hline\framedhslinecorrect\\{-1.5ex}%
   \let\endoflinesave=\\
   \let\\=\@normalcr
   \(\pboxed}%
  {\endpboxed\)%
   \framedhslinecorrect\endoflinesave{.5ex}\hline
   \endtabular
   \parskip=\belowdisplayskip\par\noindent
   \ignorespacesafterend}

\newcommand{\framedhslinecorrect}[2]%
  {#1[#2]}

\newcommand{\framedhs}{\sethscode{framedhscode}}

% The inlinehscode environment is an experimental environment
% that can be used to typeset displayed code inline.

\newenvironment{inlinehscode}%
  {\(\def\column##1##2{}%
   \let\>\undefined\let\<\undefined\let\\\undefined
   \newcommand\>[1][]{}\newcommand\<[1][]{}\newcommand\\[1][]{}%
   \def\fromto##1##2##3{##3}%
   \def\nextline{}}{\) }%

\newcommand{\inlinehs}{\sethscode{inlinehscode}}

% The joincode environment is a separate environment that
% can be used to surround and thereby connect multiple code
% blocks.

\newenvironment{joincode}%
  {\let\orighscode=\hscode
   \let\origendhscode=\endhscode
   \def\endhscode{\def\hscode{\endgroup\def\@currenvir{hscode}\\}\begingroup}
   %\let\SaveRestoreHook=\empty
   %\let\ColumnHook=\empty
   %\let\resethooks=\empty
   \orighscode\def\hscode{\endgroup\def\@currenvir{hscode}}}%
  {\origendhscode
   \global\let\hscode=\orighscode
   \global\let\endhscode=\origendhscode}%

\makeatother
\EndFmtInput
%

\begin{document}

\maketitle


\section{Introduction}

This module contains a parser for ARM 32-bit machine code. Its
purpose is to `re-compile' the machine code into a series of
haskell functions and data structures that we can manipulate
and effectively reason with. The information we extract from
the binary here will be used to aggregate the code into gadgets,
and -- together with the information extracted from the user-
provided script or specifications -- guide their compilation into
an initial population of ROP-chains.

\section{Dependencies}

First we'll need to import a handful of libraries. The most
interesting among these is Attoparsec, a fast binary-parsing
library that we're pulling from Hackage. Aux is a module of
general-purpose convenience functions, which I've written, and
the rest of the imports provide some fairly standard data types
and functions (Data.Word provides fixed width integer types,
Data.Bits gives us a few standard bitwise operations, etc.)

\begin{hscode}\SaveRestoreHook
\column{B}{@{}>{\hspre}l<{\hspost}@{}}%
\column{E}{@{}>{\hspre}l<{\hspost}@{}}%
\>[B]{}\mathbf{module}\;\Conid{ARM32}\;\mathbf{where}{}\<[E]%
\\[\blanklineskip]%
\>[B]{}\mathbf{import}\;\Conid{Aux}{}\<[E]%
\\
\>[B]{}\mathbf{import}\;\Conid{ARMCommon}{}\<[E]%
\\[\blanklineskip]%
\>[B]{}\mathbf{import}\;\Conid{\Conid{Data}.Word}{}\<[E]%
\\
\>[B]{}\mathbf{import}\;\Conid{\Conid{Data}.Bits}{}\<[E]%
\\[\blanklineskip]%
\>[B]{}\mathbf{import}\;\Conid{Numeric}\;(\Varid{showHex}){}\<[E]%
\ColumnHook
\end{hscode}\resethooks


\section{Data Types}

Here we introduce a few data types that we'll be using to
organize the information we extract from each ARM instruction,
and define a few constants.

It will be handy to fix a few constants so that we can easily
refer to a few important registers. In particular, we want to
have constant pointers to the stack pointer (SP), the link
register (LR), the programme counter (PC), and the frame pointer
(FP).

Under ordinary circumstances, the frame pointer (FP) points to
the bottom of the stack frame for the active function, and, upon
exiting the function, it is used to restore the stack pointer (SP)
to where it was before that function had its way with it.

Unlike what we see with the x86, the function calling convention
for the ARM specifies that the return address for a function be
placed in a register reserved for that purpose, the link register.
The branch-link (BL) instruction does this by performing two actions
together: it sets the programme counter (PC) to the function's entry
point, and LR to the address of the instruction where execution will
resume.

For nested function calls, however, the familiar convention of pushing
the return address onto the stack is used, and so there are typically
two different conventions for exiting a function:

(1) MOV     PC, LR     ;; copy the lr into the pc
(2) LDMIA   SP!, {PC}  ;; pop the stack into pc

\begin{hscode}\SaveRestoreHook
\column{B}{@{}>{\hspre}l<{\hspost}@{}}%
\column{3}{@{}>{\hspre}l<{\hspost}@{}}%
\column{E}{@{}>{\hspre}l<{\hspost}@{}}%
\>[3]{}\mbox{\onelinecomment  \ensuremath{\Conid{Moved}\;\Varid{to}\;\Varid{\Conid{ARMCommon}.lhs}} --}{}\<[E]%
\ColumnHook
\end{hscode}\resethooks


Compared to the motley structure of the x86 instruction set, the
ARM instruction set has a fairly regular format. There are just
14 (or 15, if we count "undefined") different layouts that the
instruction set makes use of, and so, before we can extract the
fields we need from the instruction (operation, source and desti-
nation registers, etc.).

Each layout can be recognized by its signature pattern of high
and low bits. There is not, however, a generic `layout' field
in the instruction, and so several different masks -- each isolating
different bitfields -- may need to be applied to an instruction
before we can decipher its layout. For this we will be using the
\texttt{mask} function defined in Aux.hs.

The development here is still lagging behind the developmento of the Thumb16.lhs module. In time, the layout instances will be furnished with additional fields encoding important type information about each instruction, such as its mnemonic/opcode, immediate values, etc.

\begin{hscode}\SaveRestoreHook
\column{B}{@{}>{\hspre}l<{\hspost}@{}}%
\column{3}{@{}>{\hspre}l<{\hspost}@{}}%
\column{5}{@{}>{\hspre}l<{\hspost}@{}}%
\column{43}{@{}>{\hspre}l<{\hspost}@{}}%
\column{55}{@{}>{\hspre}l<{\hspost}@{}}%
\column{E}{@{}>{\hspre}l<{\hspost}@{}}%
\>[B]{}\mathbf{data}\;\Conid{Layout}\mathrel{=}{}\<[E]%
\\
\>[B]{}\hsindent{3}{}\<[3]%
\>[3]{}\Conid{DataProc}\;\Conid{DPMnemonic}{}\<[E]%
\\
\>[B]{}\hsindent{3}{}\<[3]%
\>[3]{}\mid \Conid{Mult}{}\<[E]%
\\
\>[B]{}\hsindent{3}{}\<[3]%
\>[3]{}\mid \Conid{MultLong}{}\<[E]%
\\
\>[B]{}\hsindent{3}{}\<[3]%
\>[3]{}\mid \Conid{SingDataSwap}{}\<[E]%
\\
\>[B]{}\hsindent{3}{}\<[3]%
\>[3]{}\mid \Conid{BranchExch}{}\<[E]%
\\
\>[B]{}\hsindent{3}{}\<[3]%
\>[3]{}\mid \Conid{HalfWordDataI}{}\<[E]%
\\
\>[B]{}\hsindent{3}{}\<[3]%
\>[3]{}\mid \Conid{HalfWordDataR}{}\<[E]%
\\
\>[B]{}\hsindent{3}{}\<[3]%
\>[3]{}\mid \Conid{SingDataTrans}{}\<[E]%
\\
\>[B]{}\hsindent{3}{}\<[3]%
\>[3]{}\mid \Conid{Undef}{}\<[E]%
\\
\>[B]{}\hsindent{3}{}\<[3]%
\>[3]{}\mid \Conid{BlockDataTrans}{}\<[E]%
\\
\>[B]{}\hsindent{3}{}\<[3]%
\>[3]{}\mid \Conid{Branch}{}\<[E]%
\\
\>[B]{}\hsindent{3}{}\<[3]%
\>[3]{}\mid \Conid{CoprocDataTrans}{}\<[E]%
\\
\>[B]{}\hsindent{3}{}\<[3]%
\>[3]{}\mid \Conid{CoprocDataOp}{}\<[E]%
\\
\>[B]{}\hsindent{3}{}\<[3]%
\>[3]{}\mid \Conid{CoprocRegTrans}{}\<[E]%
\\
\>[B]{}\hsindent{3}{}\<[3]%
\>[3]{}\mid \Conid{SWI}{}\<[E]%
\\
\>[B]{}\hsindent{3}{}\<[3]%
\>[3]{}\mid \Conid{RAWDATA}\;\mathbf{deriving}\;(\Conid{Eq},\Conid{Show}){}\<[E]%
\\[\blanklineskip]%
\>[B]{}\Varid{whatLayout}\mathbin{::}\Conid{Word32}\to \Conid{Layout}{}\<[E]%
\\
\>[B]{}\Varid{whatLayout}\;\Varid{w}{}\<[E]%
\\
\>[B]{}\hsindent{3}{}\<[3]%
\>[3]{}\mid \Varid{mask}\;\Varid{w}\;\mathrm{4}\;\mathrm{8}\equiv \mathrm{9}\mathrel{\wedge}\Varid{mask}\;\Varid{w}\;\mathrm{22}\;\mathrm{28}\equiv \mathrm{0}{}\<[43]%
\>[43]{}\mathrel{=}\Conid{Mult}{}\<[E]%
\\
\>[B]{}\hsindent{3}{}\<[3]%
\>[3]{}\mid \Varid{mask}\;\Varid{w}\;\mathrm{4}\;\mathrm{8}\equiv \mathrm{9}\mathrel{\wedge}\Varid{mask}\;\Varid{w}\;\mathrm{23}\;\mathrm{28}\equiv \mathrm{1}{}\<[43]%
\>[43]{}\mathrel{=}\Conid{MultLong}{}\<[E]%
\\
\>[B]{}\hsindent{3}{}\<[3]%
\>[3]{}\mid \Varid{mask}\;\Varid{w}\;\mathrm{4}\;\mathrm{12}\equiv \mathrm{9}\mathrel{\wedge}\Varid{mask}\;\Varid{w}\;\mathrm{23}\;\mathrm{28}\equiv \mathrm{2}\mathrel{=}\Conid{SingDataSwap}{}\<[E]%
\\
\>[B]{}\hsindent{3}{}\<[3]%
\>[3]{}\mid \Varid{mask}\;\Varid{w}\;\mathrm{4}\;\mathrm{28}\equiv \mathrm{0}\;\Varid{x12fff1}{}\<[43]%
\>[43]{}\mathrel{=}\Conid{BranchExch}{}\<[E]%
\\
\>[B]{}\hsindent{3}{}\<[3]%
\>[3]{}\mid \Varid{mask}\;\Varid{w}\;\mathrm{4}\;\mathrm{5}\equiv \mathrm{1}\mathrel{\wedge}\Varid{mask}\;\Varid{w}\;\mathrm{7}\;\mathrm{12}\equiv \mathrm{1}{}\<[E]%
\\
\>[3]{}\hsindent{2}{}\<[5]%
\>[5]{}\mathrel{\wedge}\Varid{mask}\;\Varid{w}\;\mathrm{25}\;\mathrm{28}\equiv \mathrm{0}{}\<[43]%
\>[43]{}\mathrel{=}\Conid{HalfWordDataR}{}\<[E]%
\\
\>[B]{}\hsindent{3}{}\<[3]%
\>[3]{}\mid \Varid{mask}\;\Varid{w}\;\mathrm{4}\;\mathrm{5}\equiv \mathrm{1}\mathrel{\wedge}\Varid{mask}\;\Varid{w}\;\mathrm{7}\;\mathrm{12}\mathbin{>}\mathrm{1}{}\<[E]%
\\
\>[3]{}\hsindent{2}{}\<[5]%
\>[5]{}\mathrel{\wedge}\Varid{mask}\;\Varid{w}\;\mathrm{25}\;\mathrm{28}\equiv \mathrm{0}{}\<[43]%
\>[43]{}\mathrel{=}\Conid{HalfWordDataI}{}\<[E]%
\\
\>[B]{}\hsindent{3}{}\<[3]%
\>[3]{}\mid \Varid{mask}\;\Varid{w}\;\mathrm{26}\;\mathrm{28}\equiv \mathrm{1}{}\<[43]%
\>[43]{}\mathrel{=}\Conid{SingDataTrans}{}\<[E]%
\\
\>[B]{}\hsindent{3}{}\<[3]%
\>[3]{}\mid \Varid{mask}\;\Varid{w}\;\mathrm{25}\;\mathrm{28}\equiv \mathrm{3}\mathrel{\wedge}\Varid{mask}\;\Varid{w}\;\mathrm{4}\;\mathrm{5}\equiv \mathrm{1}{}\<[43]%
\>[43]{}\mathrel{=}\Conid{Undef}{}\<[E]%
\\
\>[B]{}\hsindent{3}{}\<[3]%
\>[3]{}\mid \Varid{mask}\;\Varid{w}\;\mathrm{25}\;\mathrm{28}\equiv \mathrm{4}{}\<[43]%
\>[43]{}\mathrel{=}\Conid{BlockDataTrans}{}\<[E]%
\\
\>[B]{}\hsindent{3}{}\<[3]%
\>[3]{}\mid \Varid{mask}\;\Varid{w}\;\mathrm{25}\;\mathrm{28}\equiv \mathrm{5}{}\<[43]%
\>[43]{}\mathrel{=}\Conid{Branch}{}\<[E]%
\\
\>[B]{}\hsindent{3}{}\<[3]%
\>[3]{}\mid \Varid{mask}\;\Varid{w}\;\mathrm{25}\;\mathrm{28}\equiv \mathrm{6}{}\<[43]%
\>[43]{}\mathrel{=}\Conid{CoprocDataTrans}{}\<[E]%
\\
\>[B]{}\hsindent{3}{}\<[3]%
\>[3]{}\mid \Varid{mask}\;\Varid{w}\;\mathrm{24}\;\mathrm{28}\equiv \mathrm{14}\mathrel{\wedge}\Varid{mask}\;\Varid{w}\;\mathrm{4}\;\mathrm{5}\equiv \mathrm{0}\mathrel{=}\Conid{CoprocDataOp}{}\<[E]%
\\
\>[B]{}\hsindent{3}{}\<[3]%
\>[3]{}\mid \Varid{mask}\;\Varid{w}\;\mathrm{24}\;\mathrm{28}\equiv \mathrm{14}\mathrel{\wedge}\Varid{mask}\;\Varid{w}\;\mathrm{4}\;\mathrm{5}\equiv \mathrm{1}\mathrel{=}\Conid{CoprocRegTrans}{}\<[E]%
\\
\>[B]{}\hsindent{3}{}\<[3]%
\>[3]{}\mid \Varid{mask}\;\Varid{w}\;\mathrm{24}\;\mathrm{28}\equiv \mathrm{15}{}\<[43]%
\>[43]{}\mathrel{=}\Conid{SWI}{}\<[E]%
\\
\>[B]{}\hsindent{3}{}\<[3]%
\>[3]{}\mid \Varid{mask}\;\Varid{w}\;\mathrm{25}\;\mathrm{28}\leq \mathrm{1}{}\<[43]%
\>[43]{}\mathrel{=}\Conid{DataProc}\;{}\<[55]%
\>[55]{}(\Varid{en}\mathbin{\$}\Varid{mask}\;\Varid{w}\;\mathrm{21}\;\mathrm{25}){}\<[E]%
\\[\blanklineskip]%
\>[B]{}\hsindent{3}{}\<[3]%
\>[3]{}\mid \Varid{otherwise}\mathrel{=}\Conid{RAWDATA}{}\<[E]%
\ColumnHook
\end{hscode}\resethooks

The mnemonic types here will be developed in a fashion analogous to their counterparts in the Thumb16.lhs module. 

\begin{hscode}\SaveRestoreHook
\column{B}{@{}>{\hspre}l<{\hspost}@{}}%
\column{15}{@{}>{\hspre}l<{\hspost}@{}}%
\column{E}{@{}>{\hspre}l<{\hspost}@{}}%
\>[B]{}\mathbf{data}\;\Conid{DPMnemonic}\mathrel{=}\Conid{AND}{}\<[E]%
\\
\>[B]{}\hsindent{15}{}\<[15]%
\>[15]{}\mid \Conid{EOR}{}\<[E]%
\\
\>[B]{}\hsindent{15}{}\<[15]%
\>[15]{}\mid \Conid{SUB}{}\<[E]%
\\
\>[B]{}\hsindent{15}{}\<[15]%
\>[15]{}\mid \Conid{RSB}{}\<[E]%
\\
\>[B]{}\hsindent{15}{}\<[15]%
\>[15]{}\mid \Conid{ADD}{}\<[E]%
\\
\>[B]{}\hsindent{15}{}\<[15]%
\>[15]{}\mid \Conid{ADC}{}\<[E]%
\\
\>[B]{}\hsindent{15}{}\<[15]%
\>[15]{}\mid \Conid{SBC}{}\<[E]%
\\
\>[B]{}\hsindent{15}{}\<[15]%
\>[15]{}\mid \Conid{RSC}{}\<[E]%
\\
\>[B]{}\hsindent{15}{}\<[15]%
\>[15]{}\mid \Conid{TST}{}\<[E]%
\\
\>[B]{}\hsindent{15}{}\<[15]%
\>[15]{}\mid \Conid{TEQ}{}\<[E]%
\\
\>[B]{}\hsindent{15}{}\<[15]%
\>[15]{}\mid \Conid{CMP}{}\<[E]%
\\
\>[B]{}\hsindent{15}{}\<[15]%
\>[15]{}\mid \Conid{CMN}{}\<[E]%
\\
\>[B]{}\hsindent{15}{}\<[15]%
\>[15]{}\mid \Conid{ORR}{}\<[E]%
\\
\>[B]{}\hsindent{15}{}\<[15]%
\>[15]{}\mid \Conid{MOV}{}\<[E]%
\\
\>[B]{}\hsindent{15}{}\<[15]%
\>[15]{}\mid \Conid{BIC}{}\<[E]%
\\
\>[B]{}\hsindent{15}{}\<[15]%
\>[15]{}\mid \Conid{MVN}{}\<[E]%
\\
\>[B]{}\hsindent{15}{}\<[15]%
\>[15]{}\mid \Conid{PLACEHOLDER}\;\mathbf{deriving}\;(\Conid{Enum},\Conid{Eq},\Conid{Show}){}\<[E]%
\ColumnHook
\end{hscode}\resethooks

\section{Extracting Instruction Information}

The functions defined in this section are concerned with
extracting information from machine code instructions --
what registers do they read from and write to? What operations
do they perform? and so on. The information is emitted not,
primarily, as human-readable text, but as haskell functions
and data structures.

\subsection{Layout}
The first thing we need to know about an ARM instruction is
its `layout', or instruction type. This will tell us which bits
we need to look at for the rest of the information we're after.
\subsection{Conditional Field}

Each ARM instruction has a conditional execution field, which
determines the circumstances -- the state of the flags
%% NB: it's not called "flags", look this up
register -- under which it will execute.
%% bit more on what this looks like in assembly. This field
is, conveniently, located in the same place in every layout.

\begin{hscode}\SaveRestoreHook
\column{B}{@{}>{\hspre}l<{\hspost}@{}}%
\column{3}{@{}>{\hspre}l<{\hspost}@{}}%
\column{E}{@{}>{\hspre}l<{\hspost}@{}}%
\>[B]{}\Varid{whatCond}\mathbin{::}\Conid{Word32}\to \Conid{Cond}{}\<[E]%
\\
\>[B]{}\Varid{whatCond}\;\Varid{w}\mathrel{=}{}\<[E]%
\\
\>[B]{}\hsindent{3}{}\<[3]%
\>[3]{}\Varid{toEnum}\;(\Varid{fromIntegral}\mathbin{\$}\Varid{mask}\;\Varid{w}\;\mathrm{28}\;\mathrm{32}){}\<[E]%
\ColumnHook
\end{hscode}\resethooks

\subsection{Source and Destination Registers}


\begin{hscode}\SaveRestoreHook
\column{B}{@{}>{\hspre}l<{\hspost}@{}}%
\column{3}{@{}>{\hspre}l<{\hspost}@{}}%
\column{19}{@{}>{\hspre}l<{\hspost}@{}}%
\column{22}{@{}>{\hspre}l<{\hspost}@{}}%
\column{E}{@{}>{\hspre}l<{\hspost}@{}}%
\>[B]{}\Varid{srcRegs}\mathbin{::}\Conid{Word32}\to [\mskip1.5mu \Conid{Int}\mskip1.5mu]{}\<[E]%
\\
\>[B]{}\Varid{srcRegs}\;\Varid{w}\mathrel{=}\Varid{map}\;\Varid{fromIntegral}\mathbin{\$}\mathbf{case}\;(\Varid{whatLayout}\;\Varid{w})\;\mathbf{of}{}\<[E]%
\\
\>[B]{}\hsindent{3}{}\<[3]%
\>[3]{}\Conid{DataProc}\;\anonymous {}\<[19]%
\>[19]{}\to [\mskip1.5mu \Varid{mask}\;\Varid{w}\;\mathrm{16}\;\mathrm{20}\mskip1.5mu]{}\<[E]%
\\
\>[B]{}\hsindent{3}{}\<[3]%
\>[3]{}\Conid{Mult}{}\<[19]%
\>[19]{}\to [\mskip1.5mu \Varid{mask}\;\Varid{w}\;\mathrm{12}\;\mathrm{16}\mskip1.5mu]{}\<[E]%
\\
\>[B]{}\hsindent{3}{}\<[3]%
\>[3]{}\Conid{MultLong}{}\<[19]%
\>[19]{}\to [\mskip1.5mu \Varid{mask}\;\Varid{w}\;\mathrm{8}\;\mathrm{12},\Varid{mask}\;\Varid{w}\;\mathrm{0}\;\mathrm{4}\mskip1.5mu]\mbox{\onelinecomment  check}{}\<[E]%
\\
\>[B]{}\hsindent{3}{}\<[3]%
\>[3]{}\Conid{SingDataSwap}{}\<[19]%
\>[19]{}\to [\mskip1.5mu \Varid{mask}\;\Varid{w}\;\mathrm{16}\;\mathrm{20},\Varid{mask}\;\Varid{w}\;\mathrm{0}\;\mathrm{4}\mskip1.5mu]{}\<[E]%
\\
\>[B]{}\hsindent{3}{}\<[3]%
\>[3]{}\Conid{BranchExch}{}\<[19]%
\>[19]{}\to [\mskip1.5mu \Varid{mask}\;\Varid{w}\;\mathrm{0}\;\mathrm{4}\mskip1.5mu]{}\<[E]%
\\
\>[B]{}\hsindent{3}{}\<[3]%
\>[3]{}\Conid{HalfWordDataI}{}\<[19]%
\>[19]{}\to [\mskip1.5mu \Varid{mask}\;\Varid{w}\;\mathrm{16}\;\mathrm{20}\mskip1.5mu]{}\<[E]%
\\
\>[B]{}\hsindent{3}{}\<[3]%
\>[3]{}\Conid{HalfWordDataR}{}\<[19]%
\>[19]{}\to [\mskip1.5mu \Varid{mask}\;\Varid{w}\;\mathrm{0}\;\mathrm{4},\Varid{mask}\;\Varid{w}\;\mathrm{12}\;\mathrm{16}\mskip1.5mu]{}\<[E]%
\\
\>[B]{}\hsindent{3}{}\<[3]%
\>[3]{}\Conid{SingDataTrans}{}\<[19]%
\>[19]{}\to [\mskip1.5mu \Varid{mask}\;\Varid{w}\;\mathrm{16}\;\mathrm{20}\mskip1.5mu]{}\<[E]%
\\
\>[B]{}\hsindent{3}{}\<[3]%
\>[3]{}\Conid{Undef}{}\<[19]%
\>[19]{}\to [\mskip1.5mu \mskip1.5mu]{}\<[E]%
\\
\>[B]{}\hsindent{3}{}\<[3]%
\>[3]{}\Conid{BlockDataTrans}{}\<[19]%
\>[19]{}\to \mathbf{if}\;(\Varid{testBit}\;\Varid{w}\;\mathrm{20}){}\<[E]%
\\
\>[19]{}\hsindent{3}{}\<[22]%
\>[22]{}\mathbf{then}\;[\mskip1.5mu \Varid{mask}\;\Varid{w}\;\mathrm{16}\;\mathrm{20}\mskip1.5mu]{}\<[E]%
\\
\>[19]{}\hsindent{3}{}\<[22]%
\>[22]{}\mathbf{else}\;\Varid{m\char95 block\char95 data\char95 regs}\;\Varid{w}{}\<[E]%
\\
\>[B]{}\hsindent{3}{}\<[3]%
\>[3]{}\Conid{Branch}{}\<[19]%
\>[19]{}\to [\mskip1.5mu \mskip1.5mu]{}\<[E]%
\\
\>[B]{}\hsindent{3}{}\<[3]%
\>[3]{}\Conid{CoprocDataTrans}\to [\mskip1.5mu \mskip1.5mu]{}\<[E]%
\\
\>[B]{}\hsindent{3}{}\<[3]%
\>[3]{}\Conid{CoprocDataOp}{}\<[19]%
\>[19]{}\to [\mskip1.5mu \mskip1.5mu]{}\<[E]%
\\
\>[B]{}\hsindent{3}{}\<[3]%
\>[3]{}\Conid{CoprocRegTrans}{}\<[19]%
\>[19]{}\to [\mskip1.5mu \Varid{mask}\;\Varid{w}\;\mathrm{12}\;\mathrm{16}\mskip1.5mu]{}\<[E]%
\\
\>[B]{}\hsindent{3}{}\<[3]%
\>[3]{}\Conid{SWI}{}\<[19]%
\>[19]{}\to [\mskip1.5mu \mskip1.5mu]{}\<[E]%
\\
\>[B]{}\hsindent{3}{}\<[3]%
\>[3]{}\Conid{RAWDATA}{}\<[19]%
\>[19]{}\to [\mskip1.5mu \mskip1.5mu]{}\<[E]%
\\
\>[B]{}\mbox{\onelinecomment   otherwise       -> error $ "Not yet implemented: 0x" ++ showHex w ""}{}\<[E]%
\\[\blanklineskip]%
\>[B]{}\Varid{dstRegs}\mathbin{::}\Conid{Word32}\to [\mskip1.5mu \Conid{Int}\mskip1.5mu]{}\<[E]%
\\
\>[B]{}\Varid{dstRegs}\;\Varid{w}\mathrel{=}\Varid{map}\;\Varid{fromIntegral}\mathbin{\$}\mathbf{case}\;(\Varid{whatLayout}\;\Varid{w})\;\mathbf{of}{}\<[E]%
\\
\>[B]{}\hsindent{3}{}\<[3]%
\>[3]{}\Conid{DataProc}\;\anonymous {}\<[19]%
\>[19]{}\to [\mskip1.5mu \Varid{mask}\;\Varid{w}\;\mathrm{12}\;\mathrm{16}\mskip1.5mu]{}\<[E]%
\\
\>[B]{}\hsindent{3}{}\<[3]%
\>[3]{}\Conid{Mult}{}\<[19]%
\>[19]{}\to [\mskip1.5mu \Varid{mask}\;\Varid{w}\;\mathrm{16}\;\mathrm{20}\mskip1.5mu]{}\<[E]%
\\
\>[B]{}\hsindent{3}{}\<[3]%
\>[3]{}\Conid{MultLong}{}\<[19]%
\>[19]{}\to [\mskip1.5mu \Varid{mask}\;\Varid{w}\;\mathrm{12}\;\mathrm{16},\Varid{mask}\;\Varid{w}\;\mathrm{16}\;\mathrm{20}\mskip1.5mu]{}\<[E]%
\\
\>[B]{}\hsindent{3}{}\<[3]%
\>[3]{}\Conid{SingDataSwap}{}\<[19]%
\>[19]{}\to [\mskip1.5mu \Varid{mask}\;\Varid{w}\;\mathrm{12}\;\mathrm{16}\mskip1.5mu]{}\<[E]%
\\
\>[B]{}\hsindent{3}{}\<[3]%
\>[3]{}\Conid{BranchExch}{}\<[19]%
\>[19]{}\to [\mskip1.5mu \mskip1.5mu]{}\<[E]%
\\
\>[B]{}\hsindent{3}{}\<[3]%
\>[3]{}\Conid{HalfWordDataI}{}\<[19]%
\>[19]{}\to [\mskip1.5mu \Varid{mask}\;\Varid{w}\;\mathrm{12}\;\mathrm{16}\mskip1.5mu]{}\<[E]%
\\
\>[B]{}\hsindent{3}{}\<[3]%
\>[3]{}\Conid{HalfWordDataR}{}\<[19]%
\>[19]{}\to [\mskip1.5mu \Varid{mask}\;\Varid{w}\;\mathrm{12}\;\mathrm{16}\mskip1.5mu]{}\<[E]%
\\
\>[B]{}\hsindent{3}{}\<[3]%
\>[3]{}\Conid{SingDataTrans}{}\<[19]%
\>[19]{}\to [\mskip1.5mu \Varid{mask}\;\Varid{w}\;\mathrm{12}\;\mathrm{16}\mskip1.5mu]{}\<[E]%
\\
\>[B]{}\hsindent{3}{}\<[3]%
\>[3]{}\Conid{Undef}{}\<[19]%
\>[19]{}\to [\mskip1.5mu \mskip1.5mu]{}\<[E]%
\\
\>[B]{}\hsindent{3}{}\<[3]%
\>[3]{}\Conid{BlockDataTrans}{}\<[19]%
\>[19]{}\to \mathbf{if}\;(\Varid{testBit}\;\Varid{w}\;\mathrm{20}){}\<[E]%
\\
\>[19]{}\hsindent{3}{}\<[22]%
\>[22]{}\mathbf{then}\;\Varid{m\char95 block\char95 data\char95 regs}\;\Varid{w}{}\<[E]%
\\
\>[19]{}\hsindent{3}{}\<[22]%
\>[22]{}\mathbf{else}\;[\mskip1.5mu \Varid{mask}\;\Varid{w}\;\mathrm{16}\;\mathrm{20}\mskip1.5mu]{}\<[E]%
\\
\>[B]{}\hsindent{3}{}\<[3]%
\>[3]{}\Conid{Branch}{}\<[19]%
\>[19]{}\to [\mskip1.5mu \mskip1.5mu]{}\<[E]%
\\
\>[B]{}\hsindent{3}{}\<[3]%
\>[3]{}\Conid{CoprocDataTrans}\to [\mskip1.5mu \mskip1.5mu]{}\<[E]%
\\
\>[B]{}\hsindent{3}{}\<[3]%
\>[3]{}\Conid{CoprocDataOp}{}\<[19]%
\>[19]{}\to [\mskip1.5mu \mskip1.5mu]{}\<[E]%
\\
\>[B]{}\hsindent{3}{}\<[3]%
\>[3]{}\Conid{CoprocRegTrans}{}\<[19]%
\>[19]{}\to [\mskip1.5mu \Varid{mask}\;\Varid{w}\;\mathrm{12}\;\mathrm{16}\mskip1.5mu]{}\<[E]%
\\
\>[B]{}\hsindent{3}{}\<[3]%
\>[3]{}\Conid{SWI}{}\<[19]%
\>[19]{}\to [\mskip1.5mu \mskip1.5mu]{}\<[E]%
\\
\>[B]{}\hsindent{3}{}\<[3]%
\>[3]{}\Conid{RAWDATA}{}\<[19]%
\>[19]{}\to [\mskip1.5mu \mskip1.5mu]{}\<[E]%
\\
\>[B]{}\mbox{\onelinecomment   otherwise       -> error $ "Not yet implemented: " ++ showHex w "\nGet back to work!"}{}\<[E]%
\ColumnHook
\end{hscode}\resethooks

\subsection{Operations}

Here, we will extract the actual operation that the instruction
is performing, and store it as a live haskell function.
Operations will be defined as a function of three 32-bit words
to a fourth 32-bit word.

$$ \lambda \texttt{operand_1} \texttt{operand_2} \texttt{destination} . \textit{result} $$

The reason for including the value of the destination register
in the signature is so that we can handle identity operations,
without changing type. (Some ARM instructions -- \text{CMP} for
instance -- do not alter the destination registers in any way, but
are peformed only to update the status or flag register.) %
\footnote{I have not yet implemented the flag registers or
conditional execution, in this model.}

\begin{hscode}\SaveRestoreHook
\column{B}{@{}>{\hspre}l<{\hspost}@{}}%
\column{3}{@{}>{\hspre}l<{\hspost}@{}}%
\column{5}{@{}>{\hspre}l<{\hspost}@{}}%
\column{7}{@{}>{\hspre}l<{\hspost}@{}}%
\column{9}{@{}>{\hspre}l<{\hspost}@{}}%
\column{11}{@{}>{\hspre}l<{\hspost}@{}}%
\column{14}{@{}>{\hspre}l<{\hspost}@{}}%
\column{15}{@{}>{\hspre}l<{\hspost}@{}}%
\column{26}{@{}>{\hspre}l<{\hspost}@{}}%
\column{E}{@{}>{\hspre}l<{\hspost}@{}}%
\>[B]{}\mbox{\onelinecomment  we could return a pair of Word32 values to capture the flags register}{}\<[E]%
\\
\>[B]{}\mbox{\onelinecomment  or we could simplify things by just ignoring the flags for now.}{}\<[E]%
\\
\>[B]{}\Varid{operation}\mathbin{::}\Conid{Word32}\to \Conid{Operation}{}\<[E]%
\\
\>[B]{}\Varid{operation}\;\Varid{w}\mathrel{=}\mathbf{case}\;(\Varid{whatLayout}\;\Varid{w})\;\mathbf{of}{}\<[E]%
\\
\>[B]{}\hsindent{3}{}\<[3]%
\>[3]{}\Conid{DataProc}\;\Varid{m}{}\<[15]%
\>[15]{}\to \Varid{opDP}\;\Varid{m}{}\<[E]%
\\
\>[B]{}\hsindent{3}{}\<[3]%
\>[3]{}\Conid{Mult}{}\<[14]%
\>[14]{}\to \Varid{opM}{}\<[E]%
\\
\>[B]{}\hsindent{3}{}\<[3]%
\>[3]{}\Conid{MultLong}{}\<[14]%
\>[14]{}\to \Varid{opM}{}\<[E]%
\\
\>[B]{}\hsindent{3}{}\<[3]%
\>[3]{}\Varid{otherwise}{}\<[14]%
\>[14]{}\to \Varid{error}\mathbin{\$}\text{\tt \char34 Not~yet~implemented:~0x\char34}\plus \Varid{showHex}\;\Varid{w}\;\text{\tt \char34 \char92 nBACK~TO~WORK!\char34}{}\<[E]%
\\
\>[B]{}\hsindent{3}{}\<[3]%
\>[3]{}\mathbf{where}\;\mbox{\onelinecomment t = whatLayout w}{}\<[E]%
\\
\>[3]{}\hsindent{2}{}\<[5]%
\>[5]{}\Varid{opDP}\mathbin{::}\Conid{DPMnemonic}\to \Conid{Operation}{}\<[E]%
\\
\>[3]{}\hsindent{2}{}\<[5]%
\>[5]{}\Varid{opDP}\;\Varid{mnemonic}\mathrel{=}\mathbf{case}\;\Varid{mnemonic}\;\mathbf{of}{}\<[E]%
\\
\>[5]{}\hsindent{4}{}\<[9]%
\>[9]{}\mbox{\onelinecomment  ternary: op1, op2, dst; dst ignored in most}{}\<[E]%
\\
\>[5]{}\hsindent{2}{}\<[7]%
\>[7]{}\Conid{AND}\to {}\<[15]%
\>[15]{}\lambda \Varid{a}\;\Varid{x}\;\Varid{y}\;\anonymous \to (\Varid{a},\Varid{x}\mathbin{.\&.}\Varid{y}){}\<[E]%
\\
\>[5]{}\hsindent{2}{}\<[7]%
\>[7]{}\Conid{EOR}\to {}\<[15]%
\>[15]{}\lambda \Varid{a}\;\Varid{x}\;\Varid{y}\;\anonymous \to (\Varid{a},\Varid{xor}\;\Varid{x}\;\Varid{y}){}\<[E]%
\\
\>[5]{}\hsindent{2}{}\<[7]%
\>[7]{}\Conid{SUB}\to {}\<[15]%
\>[15]{}\lambda \Varid{a}\;\Varid{x}\;\Varid{y}\;\anonymous \to (\Varid{a},\Varid{x}\mathbin{-}\Varid{y}){}\<[E]%
\\
\>[5]{}\hsindent{2}{}\<[7]%
\>[7]{}\Conid{RSB}\to {}\<[15]%
\>[15]{}\lambda \Varid{a}\;\Varid{x}\;\Varid{y}\;\anonymous \to (\Varid{a},\Varid{flip}\;(\mathbin{-})\;\Varid{x}\;\Varid{y}){}\<[E]%
\\
\>[5]{}\hsindent{2}{}\<[7]%
\>[7]{}\Conid{ADD}\to {}\<[15]%
\>[15]{}\lambda \Varid{a}\;\Varid{x}\;\Varid{y}\;\anonymous \to (\Varid{a},\Varid{x}\mathbin{+}\Varid{y}){}\<[E]%
\\
\>[5]{}\hsindent{2}{}\<[7]%
\>[7]{}\Conid{ADC}\to {}\<[15]%
\>[15]{}\lambda \Varid{a}\;\Varid{x}\;\Varid{y}\;\anonymous \to (\Varid{a},\Varid{x}\mathbin{+}\Varid{y})\mbox{\onelinecomment  PLUS CARRY}{}\<[E]%
\\
\>[5]{}\hsindent{2}{}\<[7]%
\>[7]{}\Conid{SBC}\to {}\<[15]%
\>[15]{}\lambda \Varid{a}\;\Varid{x}\;\Varid{y}\;\anonymous \to (\Varid{a},\Varid{x}\mathbin{-}\Varid{y})\mbox{\onelinecomment  PLUS CARRY - 1}{}\<[E]%
\\
\>[5]{}\hsindent{2}{}\<[7]%
\>[7]{}\Conid{RSC}\to \lambda \Varid{a}\;\Varid{x}\;\Varid{y}\;\anonymous \to (\Varid{a},\Varid{flip}\;(\mathbin{-})\;\Varid{x}\;\Varid{y})\mbox{\onelinecomment  PLUS CARRY - 1}{}\<[E]%
\\
\>[5]{}\hsindent{2}{}\<[7]%
\>[7]{}\Conid{TST}\to \lambda \Varid{a}\;\anonymous \;\anonymous \;\Varid{z}\to (\Varid{a},\Varid{z})\mbox{\onelinecomment  SET FLAGS AS WITH AND}{}\<[E]%
\\
\>[5]{}\hsindent{2}{}\<[7]%
\>[7]{}\Conid{TEQ}\to \lambda \Varid{a}\;\anonymous \;\anonymous \;\Varid{z}\to (\Varid{a},\Varid{z})\mbox{\onelinecomment  SET FLAGS AS WITH EOR}{}\<[E]%
\\
\>[5]{}\hsindent{2}{}\<[7]%
\>[7]{}\Conid{CMP}\to \lambda \Varid{a}\;\anonymous \;\anonymous \;\Varid{z}\to (\Varid{a},\Varid{z})\mbox{\onelinecomment  SET FLAGS AS WITH SUB}{}\<[E]%
\\
\>[5]{}\hsindent{2}{}\<[7]%
\>[7]{}\Conid{CMN}\to \lambda \Varid{a}\;\anonymous \;\anonymous \;\Varid{z}\to (\Varid{a},\Varid{z})\mbox{\onelinecomment  SET FLAGS AS WITH ADD}{}\<[E]%
\\
\>[5]{}\hsindent{2}{}\<[7]%
\>[7]{}\Conid{ORR}\to \lambda \Varid{a}\;\Varid{x}\;\Varid{y}\;\anonymous \to (\Varid{a},\Varid{x}\mathbin{.|.}\Varid{y}){}\<[E]%
\\
\>[5]{}\hsindent{2}{}\<[7]%
\>[7]{}\Conid{MOV}\to \lambda \Varid{a}\;\anonymous \;\Varid{y}\;\anonymous \to (\Varid{a},\Varid{y}){}\<[E]%
\\
\>[5]{}\hsindent{2}{}\<[7]%
\>[7]{}\Conid{BIC}\to \lambda \Varid{a}\;\Varid{x}\;\Varid{y}\;\anonymous \to (\Varid{a},\Varid{x}\mathbin{.\&.}(\Varid{complement}\;\Varid{y})){}\<[E]%
\\
\>[5]{}\hsindent{2}{}\<[7]%
\>[7]{}\Conid{MVN}\to \lambda \Varid{a}\;\anonymous \;\Varid{y}\;\anonymous \to (\Varid{a},\Varid{complement}\;\Varid{y}){}\<[E]%
\\
\>[3]{}\hsindent{2}{}\<[5]%
\>[5]{}\Varid{opM}\mathbin{::}\Conid{Operation}{}\<[E]%
\\
\>[3]{}\hsindent{2}{}\<[5]%
\>[5]{}\Varid{opM}\mathrel{=}\lambda \Varid{a}\;\Varid{x}\;\Varid{y}\;\anonymous \to (\Varid{a},\Varid{x}\mathbin{*}\Varid{y}){}\<[E]%
\\[\blanklineskip]%
\>[B]{}\Varid{m\char95 block\char95 data\char95 regs}\mathbin{::}\Conid{Word32}\to [\mskip1.5mu \Conid{Word32}\mskip1.5mu]{}\<[E]%
\\
\>[B]{}\Varid{m\char95 block\char95 data\char95 regs}\;\Varid{w}\mathrel{=}{}\<[E]%
\\
\>[B]{}\hsindent{3}{}\<[3]%
\>[3]{}\mathbf{let}\;\Varid{rg}\mathrel{=}\Varid{mask}\;\Varid{w}\;\mathrm{0}\;\mathrm{16}{}\<[E]%
\\
\>[B]{}\hsindent{3}{}\<[3]%
\>[3]{}\mathbf{in}\;\Varid{colR}\;\Varid{rg}\;\mathrm{0}{}\<[E]%
\\
\>[B]{}\hsindent{3}{}\<[3]%
\>[3]{}\mathbf{where}\;\Varid{colR}\;\Varid{r}\;\Varid{n}{}\<[E]%
\\
\>[3]{}\hsindent{8}{}\<[11]%
\>[11]{}\mid \Varid{n}\mathbin{>}\mathrm{15}{}\<[26]%
\>[26]{}\mathrel{=}[\mskip1.5mu \mskip1.5mu]{}\<[E]%
\\
\>[3]{}\hsindent{8}{}\<[11]%
\>[11]{}\mid \Varid{r}\mathbin{.\&.}\mathrm{1}\equiv \mathrm{1}\mathrel{=}\Varid{n}\mathbin{:}\Varid{colR}\;(\Varid{shiftR}\;\Varid{r}\;\mathrm{1})\;(\Varid{n}\mathbin{+}\mathrm{1}){}\<[E]%
\\
\>[3]{}\hsindent{8}{}\<[11]%
\>[11]{}\mid \Varid{otherwise}{}\<[26]%
\>[26]{}\mathrel{=}\Varid{colR}\;(\Varid{shiftR}\;\Varid{r}\;\mathrm{1})\;(\Varid{n}\mathbin{+}\mathrm{1}){}\<[E]%
\ColumnHook
\end{hscode}\resethooks

\end{document}
