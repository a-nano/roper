\documentclass[9pt]{beamer}

\usepackage[utf8x]{inputenc}

%\usetheme{Ampang}
%\usecolortheme{ampangcolor}
%\usecolortheme{warna}

%%% XeLaTeX engine for Ubuntu Font support
%\usepackage{xltxtra}
%\setsansfont[
%BoldFont=Ubuntu-Bold.ttf,
%ItalicFont=Ubuntu-Italic.ttf,
%BoldItalicFont=Ubuntu-BoldItalic.ttf
%%%]
%{Ubuntu-Regular.ttf}
%\setmonofont{UbuntuMono-Regular.ttf}

%%
\title{ROPER:\\A Genetic ROP-Chain Compiler Targetting Embedded
Devices}
\author{Olivia Lucca Fraser}

\institute{NIMS Lab, Dalhousie University}


\begin{document}
\begin{frame}{Summary}
\textsc{roper} is an genetic \textsc{rop}-chain compiler. It uses evolutionary
methods (natural selection) to develop specially
crafted payloads called `\textsc{rop}-chains', which are used in
cyber-attacks. These permit the attacker to gain control of a
target process's execution, while leveraging that process's
privileges. 

\vspace{4pt}
A \textsc{rop}-chain differs from a traditional shellcode attack
in that it introduces no foreign code to the target process, and
does not rely on being able to write to executable memory. 

\vspace{4pt}
Instead, it hops around memory segments that have 
already been mapped as executable, assembling them into a
sort of mosaic. This mosaic performs operations that the original
program's author never intended. 

\vspace{4pt}
\textsc{roper} spawns an entire population of such mosaics, randomly at
first, and then uses the principles of natural selection 
to breed one that accomplishes \emph{precisely} what the
attacker desires, but using means the attacker, being human,
would not have herself anticipated. 

\vspace{4pt}
In fact, \textsc{roper} has already shown itself capable of
performing subtle, adaptive tasks that \textsc{rop}-chains --
whether handmade or deterministically compiled -- have never
before been capable of, achieving over 95\,\% accuracy even when
asked to classify Iris flowers, for example. This gives us a
glimpse of the \textsc{roper}'s capacity to dynamically adapt to
problems for which it was not designed. 
\end{frame}
\end{document}

