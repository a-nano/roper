\documentclass{article}
\usepackage[utf8]{inputenc}
\usepackage{amssymb}
\usepackage{amsmath}
\usepackage{listings}
\let\oldemptyset\emptyset
\let\emptyset\varnothing
\usepackage{parskip}
\usepackage{fancyvrb}

\title{A Parser for ARM and Thumb Binaries}
\author{Olivia Lucca Fraser\\B00109376}
\date{\today}

%% ODER: format ==         = "\mathrel{==}"
%% ODER: format /=         = "\neq "
%
%
\makeatletter
\@ifundefined{lhs2tex.lhs2tex.sty.read}%
  {\@namedef{lhs2tex.lhs2tex.sty.read}{}%
   \newcommand\SkipToFmtEnd{}%
   \newcommand\EndFmtInput{}%
   \long\def\SkipToFmtEnd#1\EndFmtInput{}%
  }\SkipToFmtEnd

\newcommand\ReadOnlyOnce[1]{\@ifundefined{#1}{\@namedef{#1}{}}\SkipToFmtEnd}
\usepackage{amstext}
\usepackage{amssymb}
\usepackage{stmaryrd}
\DeclareFontFamily{OT1}{cmtex}{}
\DeclareFontShape{OT1}{cmtex}{m}{n}
  {<5><6><7><8>cmtex8
   <9>cmtex9
   <10><10.95><12><14.4><17.28><20.74><24.88>cmtex10}{}
\DeclareFontShape{OT1}{cmtex}{m}{it}
  {<-> ssub * cmtt/m/it}{}
\newcommand{\texfamily}{\fontfamily{cmtex}\selectfont}
\DeclareFontShape{OT1}{cmtt}{bx}{n}
  {<5><6><7><8>cmtt8
   <9>cmbtt9
   <10><10.95><12><14.4><17.28><20.74><24.88>cmbtt10}{}
\DeclareFontShape{OT1}{cmtex}{bx}{n}
  {<-> ssub * cmtt/bx/n}{}
\newcommand{\tex}[1]{\text{\texfamily#1}}	% NEU

\newcommand{\Sp}{\hskip.33334em\relax}


\newcommand{\Conid}[1]{\mathit{#1}}
\newcommand{\Varid}[1]{\mathit{#1}}
\newcommand{\anonymous}{\kern0.06em \vbox{\hrule\@width.5em}}
\newcommand{\plus}{\mathbin{+\!\!\!+}}
\newcommand{\bind}{\mathbin{>\!\!\!>\mkern-6.7mu=}}
\newcommand{\rbind}{\mathbin{=\mkern-6.7mu<\!\!\!<}}% suggested by Neil Mitchell
\newcommand{\sequ}{\mathbin{>\!\!\!>}}
\renewcommand{\leq}{\leqslant}
\renewcommand{\geq}{\geqslant}
\usepackage{polytable}

%mathindent has to be defined
\@ifundefined{mathindent}%
  {\newdimen\mathindent\mathindent\leftmargini}%
  {}%

\def\resethooks{%
  \global\let\SaveRestoreHook\empty
  \global\let\ColumnHook\empty}
\newcommand*{\savecolumns}[1][default]%
  {\g@addto@macro\SaveRestoreHook{\savecolumns[#1]}}
\newcommand*{\restorecolumns}[1][default]%
  {\g@addto@macro\SaveRestoreHook{\restorecolumns[#1]}}
\newcommand*{\aligncolumn}[2]%
  {\g@addto@macro\ColumnHook{\column{#1}{#2}}}

\resethooks

\newcommand{\onelinecommentchars}{\quad-{}- }
\newcommand{\commentbeginchars}{\enskip\{-}
\newcommand{\commentendchars}{-\}\enskip}

\newcommand{\visiblecomments}{%
  \let\onelinecomment=\onelinecommentchars
  \let\commentbegin=\commentbeginchars
  \let\commentend=\commentendchars}

\newcommand{\invisiblecomments}{%
  \let\onelinecomment=\empty
  \let\commentbegin=\empty
  \let\commentend=\empty}

\visiblecomments

\newlength{\blanklineskip}
\setlength{\blanklineskip}{0.66084ex}

\newcommand{\hsindent}[1]{\quad}% default is fixed indentation
\let\hspre\empty
\let\hspost\empty
\newcommand{\NB}{\textbf{NB}}
\newcommand{\Todo}[1]{$\langle$\textbf{To do:}~#1$\rangle$}

\EndFmtInput
\makeatother
%
%
%
%
%
%
% This package provides two environments suitable to take the place
% of hscode, called "plainhscode" and "arrayhscode". 
%
% The plain environment surrounds each code block by vertical space,
% and it uses \abovedisplayskip and \belowdisplayskip to get spacing
% similar to formulas. Note that if these dimensions are changed,
% the spacing around displayed math formulas changes as well.
% All code is indented using \leftskip.
%
% Changed 19.08.2004 to reflect changes in colorcode. Should work with
% CodeGroup.sty.
%
\ReadOnlyOnce{polycode.fmt}%
\makeatletter

\newcommand{\hsnewpar}[1]%
  {{\parskip=0pt\parindent=0pt\par\vskip #1\noindent}}

% can be used, for instance, to redefine the code size, by setting the
% command to \small or something alike
\newcommand{\hscodestyle}{}

% The command \sethscode can be used to switch the code formatting
% behaviour by mapping the hscode environment in the subst directive
% to a new LaTeX environment.

\newcommand{\sethscode}[1]%
  {\expandafter\let\expandafter\hscode\csname #1\endcsname
   \expandafter\let\expandafter\endhscode\csname end#1\endcsname}

% "compatibility" mode restores the non-polycode.fmt layout.

\newenvironment{compathscode}%
  {\par\noindent
   \advance\leftskip\mathindent
   \hscodestyle
   \let\\=\@normalcr
   \let\hspre\(\let\hspost\)%
   \pboxed}%
  {\endpboxed\)%
   \par\noindent
   \ignorespacesafterend}

\newcommand{\compaths}{\sethscode{compathscode}}

% "plain" mode is the proposed default.
% It should now work with \centering.
% This required some changes. The old version
% is still available for reference as oldplainhscode.

\newenvironment{plainhscode}%
  {\hsnewpar\abovedisplayskip
   \advance\leftskip\mathindent
   \hscodestyle
   \let\hspre\(\let\hspost\)%
   \pboxed}%
  {\endpboxed%
   \hsnewpar\belowdisplayskip
   \ignorespacesafterend}

\newenvironment{oldplainhscode}%
  {\hsnewpar\abovedisplayskip
   \advance\leftskip\mathindent
   \hscodestyle
   \let\\=\@normalcr
   \(\pboxed}%
  {\endpboxed\)%
   \hsnewpar\belowdisplayskip
   \ignorespacesafterend}

% Here, we make plainhscode the default environment.

\newcommand{\plainhs}{\sethscode{plainhscode}}
\newcommand{\oldplainhs}{\sethscode{oldplainhscode}}
\plainhs

% The arrayhscode is like plain, but makes use of polytable's
% parray environment which disallows page breaks in code blocks.

\newenvironment{arrayhscode}%
  {\hsnewpar\abovedisplayskip
   \advance\leftskip\mathindent
   \hscodestyle
   \let\\=\@normalcr
   \(\parray}%
  {\endparray\)%
   \hsnewpar\belowdisplayskip
   \ignorespacesafterend}

\newcommand{\arrayhs}{\sethscode{arrayhscode}}

% The mathhscode environment also makes use of polytable's parray 
% environment. It is supposed to be used only inside math mode 
% (I used it to typeset the type rules in my thesis).

\newenvironment{mathhscode}%
  {\parray}{\endparray}

\newcommand{\mathhs}{\sethscode{mathhscode}}

% texths is similar to mathhs, but works in text mode.

\newenvironment{texthscode}%
  {\(\parray}{\endparray\)}

\newcommand{\texths}{\sethscode{texthscode}}

% The framed environment places code in a framed box.

\def\codeframewidth{\arrayrulewidth}
\RequirePackage{calc}

\newenvironment{framedhscode}%
  {\parskip=\abovedisplayskip\par\noindent
   \hscodestyle
   \arrayrulewidth=\codeframewidth
   \tabular{@{}|p{\linewidth-2\arraycolsep-2\arrayrulewidth-2pt}|@{}}%
   \hline\framedhslinecorrect\\{-1.5ex}%
   \let\endoflinesave=\\
   \let\\=\@normalcr
   \(\pboxed}%
  {\endpboxed\)%
   \framedhslinecorrect\endoflinesave{.5ex}\hline
   \endtabular
   \parskip=\belowdisplayskip\par\noindent
   \ignorespacesafterend}

\newcommand{\framedhslinecorrect}[2]%
  {#1[#2]}

\newcommand{\framedhs}{\sethscode{framedhscode}}

% The inlinehscode environment is an experimental environment
% that can be used to typeset displayed code inline.

\newenvironment{inlinehscode}%
  {\(\def\column##1##2{}%
   \let\>\undefined\let\<\undefined\let\\\undefined
   \newcommand\>[1][]{}\newcommand\<[1][]{}\newcommand\\[1][]{}%
   \def\fromto##1##2##3{##3}%
   \def\nextline{}}{\) }%

\newcommand{\inlinehs}{\sethscode{inlinehscode}}

% The joincode environment is a separate environment that
% can be used to surround and thereby connect multiple code
% blocks.

\newenvironment{joincode}%
  {\let\orighscode=\hscode
   \let\origendhscode=\endhscode
   \def\endhscode{\def\hscode{\endgroup\def\@currenvir{hscode}\\}\begingroup}
   %\let\SaveRestoreHook=\empty
   %\let\ColumnHook=\empty
   %\let\resethooks=\empty
   \orighscode\def\hscode{\endgroup\def\@currenvir{hscode}}}%
  {\origendhscode
   \global\let\hscode=\orighscode
   \global\let\endhscode=\origendhscode}%

\makeatother
\EndFmtInput
%

\begin{document}



\begin{hscode}\SaveRestoreHook
\column{B}{@{}>{\hspre}l<{\hspost}@{}}%
\column{26}{@{}>{\hspre}l<{\hspost}@{}}%
\column{33}{@{}>{\hspre}l<{\hspost}@{}}%
\column{E}{@{}>{\hspre}l<{\hspost}@{}}%
\>[B]{}\mathbf{module}\;\Conid{ARMParser}\;\mathbf{where}{}\<[E]%
\\
\>[B]{}\mathbf{import}\;\Conid{Aux}{}\<[E]%
\\
\>[B]{}\mathbf{import}\;\Conid{ARMCommon}{}\<[E]%
\\
\>[B]{}\mathbf{import}\;\Varid{qualified}\;\Conid{Thumb16}\;\Varid{as}\;\Conid{Th}{}\<[33]%
\>[33]{}\mbox{\onelinecomment  to avoid namespace conflicts}{}\<[E]%
\\
\>[B]{}\mathbf{import}\;\Varid{qualified}\;\Conid{ARM32}\;{}\<[26]%
\>[26]{}\Varid{as}\;\Conid{Ar}{}\<[E]%
\\
\>[B]{}\mathbf{import}\;\Conid{\Conid{Control}.Applicative}{}\<[E]%
\\
\>[B]{}\mathbf{import}\;\Varid{qualified}\;\Conid{\Conid{Data}.List}\;\Varid{as}\;\Conid{L}{}\<[E]%
\\
\>[B]{}\mathbf{import}\;\Varid{qualified}\;\Conid{\Conid{Data}.ByteString}\;\Varid{as}\;\Conid{B}{}\<[E]%
\\
\>[B]{}\mathbf{import}\;\Conid{\Conid{Data}.\Conid{Attoparsec}.ByteString}{}\<[E]%
\\
\>[B]{}\mathbf{import}\;\Conid{\Conid{Data}.\Conid{Attoparsec}.Binary}{}\<[E]%
\\
\>[B]{}\mathbf{import}\;\Conid{\Conid{Data}.Word}{}\<[E]%
\\
\>[B]{}\mathbf{import}\;\Conid{\Conid{Data}.Bits}{}\<[E]%
\\
\>[B]{}\mathbf{import}\;\Conid{Numeric}\;(\Varid{showHex}){}\<[E]%
\ColumnHook
\end{hscode}\resethooks
\section{Parser}


Here's where we do the actual parsing. Since machine code has no real
grammar to speak of, beyond the layout structure that we've already
written several functions to pull apart, this part is almost trivial.

The control structures here are furnished by Attoparsec, a popular parsing library that deals especially well with raw binary data. Attoparsec has many resources for handling syntax on the macro level -- the relational structure obtaining between separate instructions -- but since we are dealing with fixed-width (RISC) instructions, there isn't much for it to do here. The complexities of the grammar are all situated at the 'micro' level, inside each instruction. While we could have pressed Attoparsec into service in that domain, it was just as simple to perform
 the analysis with a bit of pattern matching and bit twisting. 

\section {Endianness}

The ARM architecture is bi-endian, meaning that it can operate in either big-endian or little-endian mode. This choice is controled by a flag bit in the CPSR register (the 'E' flag). For the time being, we'll be focussing on little-endian code, and will hardcode this into our parser. In the interest of maintaining flexibility, however, we will hardcode it as delicately as possible, so that in the future we will be able to easily parameterize this option.

\begin{hscode}\SaveRestoreHook
\column{B}{@{}>{\hspre}l<{\hspost}@{}}%
\column{3}{@{}>{\hspre}l<{\hspost}@{}}%
\column{10}{@{}>{\hspre}l<{\hspost}@{}}%
\column{E}{@{}>{\hspre}l<{\hspost}@{}}%
\>[B]{}\mathbf{data}\;\Conid{Endian}\mathrel{=}\Conid{Big}\mid \Conid{Little}{}\<[E]%
\\
\>[B]{}\Varid{endian}\mathrel{=}\Conid{Little}{}\<[E]%
\\[\blanklineskip]%
\>[B]{}\Varid{anyWord16}\mathrel{=}\mathbf{case}\;\Varid{endian}\;\mathbf{of}{}\<[E]%
\\
\>[B]{}\hsindent{3}{}\<[3]%
\>[3]{}\Conid{Big}{}\<[10]%
\>[10]{}\to \Varid{anyWord16be}{}\<[E]%
\\
\>[B]{}\hsindent{3}{}\<[3]%
\>[3]{}\Conid{Little}\to \Varid{anyWord16le}{}\<[E]%
\\[\blanklineskip]%
\>[B]{}\Varid{anyWord32}\mathrel{=}\mathbf{case}\;\Varid{endian}\;\mathbf{of}{}\<[E]%
\\
\>[B]{}\hsindent{3}{}\<[3]%
\>[3]{}\Conid{Big}{}\<[10]%
\>[10]{}\to \Varid{anyWord32be}{}\<[E]%
\\
\>[B]{}\hsindent{3}{}\<[3]%
\>[3]{}\Conid{Little}\to \Varid{anyWord32le}{}\<[E]%
\\[\blanklineskip]%
\>[B]{}\Varid{anyWord64}\mathrel{=}\mathbf{case}\;\Varid{endian}\;\mathbf{of}{}\<[E]%
\\
\>[B]{}\hsindent{3}{}\<[3]%
\>[3]{}\Conid{Big}{}\<[10]%
\>[10]{}\to \Varid{anyWord64be}{}\<[E]%
\\
\>[B]{}\hsindent{3}{}\<[3]%
\>[3]{}\Conid{Little}\to \Varid{anyWord64le}{}\<[E]%
\ColumnHook
\end{hscode}\resethooks

\section {Some Auxiliary Types}

Since we will be pulling in code from both the Thumb and ARM modules here, modules which largely parallel one another, it will be helpful to have a handful of auxiliary types that can be used to braid together the two instruction modes.

We will also introduce the Inst type, which will assemble all the information we have pulled from each instruction in a convenient record type (a record is the Haskellian counterpart to C's structs).

\begin{hscode}\SaveRestoreHook
\column{B}{@{}>{\hspre}l<{\hspost}@{}}%
\column{3}{@{}>{\hspre}l<{\hspost}@{}}%
\column{4}{@{}>{\hspre}l<{\hspost}@{}}%
\column{5}{@{}>{\hspre}l<{\hspost}@{}}%
\column{10}{@{}>{\hspre}l<{\hspost}@{}}%
\column{11}{@{}>{\hspre}l<{\hspost}@{}}%
\column{13}{@{}>{\hspre}l<{\hspost}@{}}%
\column{14}{@{}>{\hspre}l<{\hspost}@{}}%
\column{E}{@{}>{\hspre}l<{\hspost}@{}}%
\>[B]{}\mathbf{data}\;\Conid{Layout'}\mathrel{=}\Conid{Thumb}\;\Conid{\Conid{Th}.Layout}\mid \Conid{ARM}\;\Conid{\Conid{Ar}.Layout}\;\mathbf{deriving}\;(\Conid{Show},\Conid{Eq}){}\<[E]%
\\[\blanklineskip]%
\>[B]{}\mathbf{data}\;\Conid{Raw}\mathrel{=}\Conid{W16}\;\Conid{Word16}\mid \Conid{W32}\;\Conid{Word32}\;\mathbf{deriving}\;(\Conid{Show},\Conid{Eq}){}\<[E]%
\\
\>[B]{}\mbox{\onelinecomment instance Enum Raw where}{}\<[E]%
\\
\>[B]{}\mbox{\onelinecomment  fromEnum r = fromEnum $ fromRaw r}{}\<[E]%
\\
\>[B]{}\mbox{\onelinecomment  toEnum r = W32 (fromEnum r)}{}\<[E]%
\\
\>[B]{}\Varid{fromRaw32}\mathbin{::}\Conid{Raw}\to \Conid{Word32}{}\<[E]%
\\
\>[B]{}\Varid{fromRaw16}\mathbin{::}\Conid{Raw}\to \Conid{Word16}{}\<[E]%
\\
\>[B]{}\Varid{fromRaw16}\;(\Conid{W16}\;\Varid{w})\mathrel{=}\Varid{w}{}\<[E]%
\\
\>[B]{}\Varid{fromRaw32}\;(\Conid{W32}\;\Varid{w})\mathrel{=}\Varid{w}{}\<[E]%
\\[\blanklineskip]%
\>[B]{}\mathbf{data}\;\Conid{Mode}\mathrel{=}\Conid{ArmMode}\mid \Conid{ThumbMode}\;\mathbf{deriving}\;(\Conid{Eq},\Conid{Show},\Conid{Enum}){}\<[E]%
\\[\blanklineskip]%
\>[B]{}\mathbf{data}\;\Conid{Inst}\mathrel{=}\Conid{Inst}\;\{\mskip1.5mu {}\<[E]%
\\
\>[B]{}\hsindent{4}{}\<[4]%
\>[4]{}\Varid{iRaw}{}\<[10]%
\>[10]{}\mathbin{::}\Conid{Raw}{}\<[E]%
\\
\>[B]{}\hsindent{3}{}\<[3]%
\>[3]{},\Varid{iLay}{}\<[10]%
\>[10]{}\mathbin{::}\Conid{Layout'}{}\<[E]%
\\
\>[B]{}\hsindent{3}{}\<[3]%
\>[3]{},\Varid{iSrc}{}\<[10]%
\>[10]{}\mathbin{::}[\mskip1.5mu \Conid{Int}\mskip1.5mu]{}\<[E]%
\\
\>[B]{}\hsindent{3}{}\<[3]%
\>[3]{},\Varid{iDst}{}\<[10]%
\>[10]{}\mathbin{::}[\mskip1.5mu \Conid{Int}\mskip1.5mu]{}\<[E]%
\\
\>[B]{}\hsindent{3}{}\<[3]%
\>[3]{},\Varid{iCnd}{}\<[10]%
\>[10]{}\mathbin{::}\Conid{Cond}{}\<[E]%
\\
\>[B]{}\hsindent{3}{}\<[3]%
\>[3]{},\Varid{iOp}{}\<[10]%
\>[10]{}\mathbin{::}\Conid{Operation}{}\<[E]%
\\
\>[B]{}\hsindent{3}{}\<[3]%
\>[3]{}\mskip1.5mu\}{}\<[E]%
\\[\blanklineskip]%
\>[B]{}\mathbf{instance}\;\Conid{Eq}\;\Conid{Inst}\;\mathbf{where}{}\<[E]%
\\
\>[B]{}\hsindent{3}{}\<[3]%
\>[3]{}\Varid{x}\equiv \Varid{y}{}\<[11]%
\>[11]{}\mathrel{=}{}\<[14]%
\>[14]{}((\Varid{iRaw}\;\Varid{x})\equiv (\Varid{iRaw}\;\Varid{y})){}\<[E]%
\\[\blanklineskip]%
\>[B]{}\mathbf{instance}\;\Conid{Show}\;\Conid{Inst}\;\mathbf{where}{}\<[E]%
\\
\>[B]{}\hsindent{3}{}\<[3]%
\>[3]{}\Varid{show}\;\Varid{x}{}\<[11]%
\>[11]{}\mathrel{=}(\Varid{show}\mathbin{\$}\Varid{iRaw}\;\Varid{x})\plus \text{\tt \char34 :~\char34}{}\<[E]%
\\
\>[11]{}\hsindent{2}{}\<[13]%
\>[13]{}\plus \text{\tt \char34 ~\char34}\plus (\Varid{stringy}\;\Varid{iSrc})\plus \text{\tt \char34 ~->\char34}\plus (\Varid{stringy}\;\Varid{iDst}){}\<[E]%
\\
\>[11]{}\hsindent{2}{}\<[13]%
\>[13]{}\plus \text{\tt \char34 ~~(\char34}\plus (\Varid{show}\mathbin{\$}\Varid{iLay}\;\Varid{x})\plus \text{\tt \char34 )\char34}\plus \text{\tt \char34 \char92 n\char34}{}\<[E]%
\\
\>[3]{}\hsindent{2}{}\<[5]%
\>[5]{}\mathbf{where}\;\Varid{stringy}\;\Varid{field}\mathrel{=}{}\<[E]%
\\
\>[5]{}\hsindent{8}{}\<[13]%
\>[13]{}(\Varid{foldl}\;(\lambda \Varid{a}\;\Varid{b}\to \Varid{a}\plus \text{\tt \char34 ~\char34}\plus \Varid{b})\;[\mskip1.5mu \mskip1.5mu]\;(\Varid{fmap}\;\Varid{show}\mathbin{\$}\Varid{field}\;\Varid{x})){}\<[E]%
\\
\>[B]{}\mbox{\onelinecomment  Some mnemonics we'll be using for DataProc instructions.}{}\<[E]%
\\
\>[B]{}\mbox{\onelinecomment  Note that a different system will be needed for the}{}\<[E]%
\\
\>[B]{}\mbox{\onelinecomment  other layouts.}{}\<[E]%
\ColumnHook
\end{hscode}\resethooks

\section {Parsing Functions}

Here, we finally introduce our parsing functions for Thumb and ARM, respectively.

\begin{hscode}\SaveRestoreHook
\column{B}{@{}>{\hspre}l<{\hspost}@{}}%
\column{3}{@{}>{\hspre}l<{\hspost}@{}}%
\column{5}{@{}>{\hspre}l<{\hspost}@{}}%
\column{6}{@{}>{\hspre}l<{\hspost}@{}}%
\column{11}{@{}>{\hspre}l<{\hspost}@{}}%
\column{E}{@{}>{\hspre}l<{\hspost}@{}}%
\>[B]{}\mbox{\onelinecomment  The instruction parser, itself}{}\<[E]%
\\
\>[B]{}\Varid{thumbInst}\mathbin{::}\Conid{Parser}\;\Conid{Inst}{}\<[E]%
\\
\>[B]{}\Varid{thumbInst}\mathrel{=}\mathbf{do}{}\<[E]%
\\
\>[B]{}\hsindent{3}{}\<[3]%
\>[3]{}\Varid{w}\leftarrow \Varid{anyWord16}{}\<[E]%
\\
\>[B]{}\hsindent{3}{}\<[3]%
\>[3]{}\Varid{pure}\mathbin{\$}\Conid{Inst}\;\{\mskip1.5mu {}\<[E]%
\\
\>[3]{}\hsindent{3}{}\<[6]%
\>[6]{}\Varid{iRaw}\mathrel{=}\Conid{W16}\;\Varid{w}{}\<[E]%
\\
\>[3]{}\hsindent{2}{}\<[5]%
\>[5]{},\Varid{iLay}\mathrel{=}\Conid{Thumb}\mathbin{\$}\Varid{\Conid{Th}.whatLayout}\;\Varid{w}{}\<[E]%
\\
\>[3]{}\hsindent{2}{}\<[5]%
\>[5]{},\Varid{iSrc}\mathrel{=}\Varid{\Conid{Th}.srcRegs}\;\Varid{w}{}\<[E]%
\\
\>[3]{}\hsindent{2}{}\<[5]%
\>[5]{},\Varid{iDst}\mathrel{=}\Varid{\Conid{Th}.dstRegs}\;\Varid{w}{}\<[E]%
\\
\>[3]{}\hsindent{2}{}\<[5]%
\>[5]{},\Varid{iCnd}\mathrel{=}\bot {}\<[E]%
\\
\>[3]{}\hsindent{2}{}\<[5]%
\>[5]{},\Varid{iOp}{}\<[11]%
\>[11]{}\mathrel{=}\Varid{\Conid{Th}.operation}\;\Varid{w}{}\<[E]%
\\
\>[3]{}\hsindent{2}{}\<[5]%
\>[5]{}\mskip1.5mu\}{}\<[E]%
\\[\blanklineskip]%
\>[B]{}\Varid{armInst}\mathbin{::}\Conid{Parser}\;\Conid{Inst}{}\<[E]%
\\
\>[B]{}\Varid{armInst}\mathrel{=}\mathbf{do}{}\<[E]%
\\
\>[B]{}\hsindent{3}{}\<[3]%
\>[3]{}\Varid{w}\leftarrow \Varid{anyWord32}{}\<[E]%
\\
\>[B]{}\hsindent{3}{}\<[3]%
\>[3]{}\Varid{pure}\mathbin{\$}\Conid{Inst}\;\{\mskip1.5mu {}\<[E]%
\\
\>[3]{}\hsindent{3}{}\<[6]%
\>[6]{}\Varid{iRaw}\mathrel{=}\Conid{W32}\;\Varid{w}{}\<[E]%
\\
\>[3]{}\hsindent{2}{}\<[5]%
\>[5]{},\Varid{iLay}\mathrel{=}\Conid{ARM}\mathbin{\$}\Varid{\Conid{Ar}.whatLayout}\;\Varid{w}{}\<[E]%
\\
\>[3]{}\hsindent{2}{}\<[5]%
\>[5]{},\Varid{iSrc}\mathrel{=}\Varid{\Conid{Ar}.srcRegs}\;\Varid{w}{}\<[E]%
\\
\>[3]{}\hsindent{2}{}\<[5]%
\>[5]{},\Varid{iDst}\mathrel{=}\Varid{\Conid{Ar}.dstRegs}\;\Varid{w}{}\<[E]%
\\
\>[3]{}\hsindent{2}{}\<[5]%
\>[5]{},\Varid{iCnd}\mathrel{=}\bot {}\<[E]%
\\
\>[3]{}\hsindent{2}{}\<[5]%
\>[5]{},\Varid{iOp}{}\<[11]%
\>[11]{}\mathrel{=}\Varid{\Conid{Ar}.operation}\;\Varid{w}{}\<[E]%
\\
\>[3]{}\hsindent{2}{}\<[5]%
\>[5]{}\mskip1.5mu\}{}\<[E]%
\ColumnHook
\end{hscode}\resethooks

Now we just need to pull this together with a parser that repeatedly applies one or the other until it exhausts the input.

Since we are not emulating the code here, we have no \emph{a priori} way of knowing whether we should be reading the code as Thumb or as ARM. This is, itself, an interesting feature of the ARM/Thumb instruction set, from the point of view of the attacker: heedless of the 'intentions' of the programmer or compiler, so long as she is able to manipulate the $T$ flag in the CPSR, the attacker is free to interpret any executable data as \emph{either} Thumb or ARM, which coexist in a sort of virtual palimpsest in the code -- so long, that is, as the code can be properly parsed.

\begin{hscode}\SaveRestoreHook
\column{B}{@{}>{\hspre}l<{\hspost}@{}}%
\column{E}{@{}>{\hspre}l<{\hspost}@{}}%
\>[B]{}\Varid{instructions}\mathbin{::}\Conid{Mode}\to \Conid{Parser}\;[\mskip1.5mu \Conid{Inst}\mskip1.5mu]{}\<[E]%
\\
\>[B]{}\Varid{instructions}\;\Conid{ArmMode}\mathrel{=}\Varid{many}\;\Varid{armInst}{}\<[E]%
\\
\>[B]{}\Varid{instructions}\;\Conid{ThumbMode}\mathrel{=}\Varid{many}\;\Varid{thumbInst}{}\<[E]%
\ColumnHook
\end{hscode}\resethooks

\section{Testing functions}

This bit here is solely for testing purposes, to see if our parser is able to process a chunk of real-world data (pulled from the .text section of ldconfig.real, a statically compiled binary, found in the bowels of a Raspberry Pi's Debian installation, whose day job is to create, update, and remove symbolic links for shared library object files, enlisted here only because it supplies a large amount of statically compiled ARM code).

\begin{hscode}\SaveRestoreHook
\column{B}{@{}>{\hspre}l<{\hspost}@{}}%
\column{3}{@{}>{\hspre}l<{\hspost}@{}}%
\column{10}{@{}>{\hspre}l<{\hspost}@{}}%
\column{28}{@{}>{\hspre}l<{\hspost}@{}}%
\column{54}{@{}>{\hspre}l<{\hspost}@{}}%
\column{E}{@{}>{\hspre}l<{\hspost}@{}}%
\>[B]{}\mbox{\onelinecomment  the text section of an ARM Elf binary, extracted with dd}{}\<[E]%
\\
\>[B]{}\mbox{\onelinecomment  just to tide us over until the Elf header parser is written}{}\<[E]%
\\
\>[B]{}\Varid{textpath}\mathrel{=}\text{\tt \char34 /home/oblivia/Projects/roper-stack/bins/arm/ldconfig.text\char34}{}\<[E]%
\\[\blanklineskip]%
\>[B]{}\Varid{main}\mathbin{::}\Conid{IO}\;(){}\<[E]%
\\
\>[B]{}\Varid{main}\mathrel{=}\mathbf{do}{}\<[E]%
\\
\>[B]{}\hsindent{3}{}\<[3]%
\>[3]{}\Varid{text}{}\<[10]%
\>[10]{}\leftarrow \Varid{\Conid{B}.readFile}\;\Varid{textpath}{}\<[E]%
\\
\>[B]{}\hsindent{3}{}\<[3]%
\>[3]{}\Varid{putStrLn}\;\text{\tt \char34 ====================================================\char34}{}\<[E]%
\\
\>[B]{}\hsindent{3}{}\<[3]%
\>[3]{}\Varid{putStrLn}\;\text{\tt \char34 ~~~~~~~~~~~~~~~~~~~Arm~Mode\char34}{}\<[E]%
\\
\>[B]{}\hsindent{3}{}\<[3]%
\>[3]{}\Varid{putStrLn}\;\text{\tt \char34 ====================================================\char34}{}\<[E]%
\\
\>[B]{}\hsindent{3}{}\<[3]%
\>[3]{}\mathbf{let}\;\Varid{aparsed}\mathrel{=}\Varid{parseOnly}\;{}\<[28]%
\>[28]{}(\Varid{instructions}\;\Conid{ArmMode})\mathbin{\$}{}\<[54]%
\>[54]{}\Varid{\Conid{B}.take}\;\mathrm{0}\;\Varid{x200}\;\Varid{text}{}\<[E]%
\\
\>[B]{}\hsindent{3}{}\<[3]%
\>[3]{}\Varid{print}\;\Varid{aparsed}{}\<[E]%
\\
\>[B]{}\hsindent{3}{}\<[3]%
\>[3]{}\Varid{putStrLn}\;\text{\tt \char34 ====================================================\char34}{}\<[E]%
\\
\>[B]{}\hsindent{3}{}\<[3]%
\>[3]{}\Varid{putStrLn}\;\text{\tt \char34 ~~~~~~~~~~~~~~~~~~~Thumb~Mode\char34}{}\<[E]%
\\
\>[B]{}\hsindent{3}{}\<[3]%
\>[3]{}\Varid{putStrLn}\;\text{\tt \char34 ====================================================\char34}{}\<[E]%
\\
\>[B]{}\hsindent{3}{}\<[3]%
\>[3]{}\mathbf{let}\;\Varid{tparsed}\mathrel{=}\Varid{parseOnly}\;{}\<[28]%
\>[28]{}(\Varid{instructions}\;\Conid{ThumbMode})\mathbin{\$}\Varid{\Conid{B}.take}\;\mathrm{0}\;\Varid{x200}\;\Varid{text}{}\<[E]%
\\
\>[B]{}\hsindent{3}{}\<[3]%
\>[3]{}\Varid{print}\;\Varid{tparsed}{}\<[E]%
\\
\>[B]{}\mbox{\onelinecomment  find some pure Thumb code to test Thumb mode with (TODO)}{}\<[E]%
\ColumnHook
\end{hscode}\resethooks

\end{document}

